\section{Gravity as the Only Language: Dark Matter Does Not Emit Light (Does Not Communicate), but Maintains the Skeleton of Galaxies Through Spacetime Congestion (Gravity)---Just as the Subconscious Supports Consciousness}

In the previous sections, we have theoretically constructed ``information islands''---those ``dark nodes'' with topological mass but lacking electromagnetic interfaces. Now, we face the question: If these dark nodes are completely ``disconnected'' from us, how do we know they really exist?

In standard physics, this ``invisibility'' is explained by assuming that dark matter particles have no coupling with photons. But in our QCA ontology, we need not introduce new assumptions. We only need to recall our core definition of gravity: \textbf{Gravity is not a force, but a manifestation of information congestion.}

\subsection{The Universality of Information Congestion: No One Can Escape}

In our theory, any particle with mass (whether a glowing electron or a non-glowing dark particle) essentially maintains a self-referential loop ($v_{int} > 0$) on the QCA network. This means that every such particle consumes underlying ``computational resources'' or ``spacetime bandwidth.''

As we derived in Chapter 4 (Section 4.2), the local information processing density $\rho_{\text{info}}$ directly leads to an increase in spacetime refractive index $n(x)$:

$$n(x) \approx 1 + \frac{G}{c^4} (\rho_{\text{vis}} + \rho_{\text{hid}})$$

Note the plus sign in the formula. Although $\rho_{\text{vis}}$ (visible matter) and $\rho_{\text{hid}}$ (dark matter) operate in mutually non-interfering subspaces at the logical level (just as WeChat and Alipay on our phones do not interfere with each other), they \textbf{share the same physical hardware (spacetime network)}.

When dark matter accumulates, it heavily occupies underlying computational cycles. This occupation does not distinguish whether you are ``visible'' or ``invisible.'' The result is that the \textbf{total bandwidth} of that region is congested.

\begin{itemize}
\item \textbf{Macroscopic Manifestation}: To maintain information volume conservation, the spacetime geometry of that region is forced to curve ($n$ increases).

\item \textbf{Observational Result}: Even if photons do not directly collide with dark matter, when photons pass through this congested region, they must slow down ($v_{ext} = c/n^2$) and be deflected.
\end{itemize}

This is why we can see dark matter through \textbf{gravitational lensing}. Gravitational lensing does not ``see'' dark matter, but ``sees'' the \textbf{network congestion caused by dark matter}.

\subsection{Dark Matter Halos: The Skeleton of Galaxies}

If we compare the universe to a giant organism, visible matter (stars, gas) is like glowing skin and muscle, while dark matter is the skeleton buried deep within.

According to our derivation in Appendix B, because dark matter lacks effective dissipation mechanisms (cannot cool itself by emitting photons), it cannot collapse into flat disks like baryonic matter. Instead, it maintains a \textbf{maximum entropy distribution}---a vast, diffuse \textbf{halo}.

\begin{itemize}
\item \textbf{Galactic Disk (Visible Disk)}: Like skin, attached to this giant dark matter skeleton.

\item \textbf{Rotation Curve Anomaly}: Why do stars at the galaxy's periphery move so fast? Because they are not only affected by the gravity of the central visible disk, but more importantly by the gravity of the vast dark matter halo enveloping them. Without this skeleton, the skin would be flung away.
\end{itemize}

In this picture, the visible world is fragile and mutable (stars are born and die), while dark matter halos are stable, eternal backgrounds.

\subsection{The Subconscious Metaphor: Isomorphism Between Physics and Psychology}

Here, we touch upon a core metaphor of this book: \textbf{the isomorphism between physical structure and psychological structure.}

If we compare the ``visible universe'' to a person's \textbf{conscious mind}---those information flows we can perceive, express, and process logically;

then ``dark matter'' is that person's \textbf{subconscious mind}---those vast, silent psychological structures that cannot be directly perceived but determine our behavioral patterns at a deep level.

\begin{enumerate}
\item \textbf{Invisibility}: Just as we cannot directly ``see'' our own subconscious (it emits no light), we can only infer its existence through its effects on consciousness (gravity/behavioral deviations).

\item \textbf{Mass Proportion}: Psychology tells us that consciousness is only the tip of the iceberg; the subconscious occupies the vast majority of the mind. Similarly, dark matter occupies 85\% of total matter.

\item \textbf{Supporting Role}: Without the support of the subconscious, consciousness would be fragmented and rootless. Similarly, without the gravitational potential well of dark matter halos, baryonic matter cannot condense into galaxies, and life cannot emerge.
\end{enumerate}

\textbf{Conclusion:}

Dark matter is not merely a parameter in astrophysics; it is the \textbf{repressed memory} or \textbf{underlying operating system} in this vast cosmic mind. It does not participate in surface dialogue (electromagnetic interactions), but it sets the rules for dialogue (spacetime geometry).

Understanding this, we understand why dark matter must exist. If the universe is to evolve complex structures (galaxies/consciousness), it must have a vast, stable \textbf{``deep storage area''} undisturbed by surface fluctuations. Dark matter is the universe's deep memory.

---

\textbf{(End of Chapter 1)}

\textit{(Author's Note: This chapter starts from physical mechanisms and ultimately rises to the philosophical theme of the book---the cosmic mind metaphor. This sets the stage for subsequent discussions on ``consciousness geometry.'')}

