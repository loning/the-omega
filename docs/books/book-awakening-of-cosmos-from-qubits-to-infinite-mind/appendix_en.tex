\chapter*{Appendix}
\addcontentsline{toc}{chapter}{Appendix}

\section*{Appendix A: Technical Details of Consciousness Models}

This appendix aims to transform qualitative descriptions of ``consciousness'' in the main text (such as $M_I$, free energy, entanglement) into computable mathematical forms. Although precise calculation of these quantities is currently impossible for complex systems like the human brain, it is necessary to provide theoretical definitions.

\subsection*{A.1 QCA Definition of Integrated Information $\Phi$}

In Sections 2.3 and 8.2 of the main text, we referenced Giulio Tononi's Integrated Information Theory (IIT). Under the QCA framework, $\Phi$ value corresponds to the \textbf{Irreducibility} of network entanglement structure.

\textbf{Definition A.1 (Causal Partition and Effective Information)}:

Let the current state of QCA system $S$ be $s_t$.

Consider partitioning the system into two parts $A$ and $B$ (bipartition), cutting all causal connections between them (set connection field $U_{AB} = \mathbb{I}$).

Let the partitioned system evolution produce state $s'_{t+1}$, while the original system produces $s_{t+1}$.

The distance between these two future states (usually measured by Kullback-Leibler divergence or Wasserstein distance) measures the ``information loss'' caused by this partition, denoted $\varphi(A, B)$.

\textbf{Definition A.2 (Integrated Information $\Phi$)}:

The system's $\Phi$ value is defined as the information loss value corresponding to the partition causing \textbf{minimum} information loss among all possible bipartitions (Minimum Information Partition, MIP).

$$\Phi(S) = \min_{\{A, B\}} \varphi(A, B)$$

\begin{itemize}
\item \textbf{Physical Meaning}: $\Phi$ measures the degree to which the system is a ``whole.'' If $\Phi=0$, the system can be decomposed into two independent subsystems without loss (like scattered sand). If $\Phi$ is large, internal connections are indivisible (like an electron or a consciousness).
\end{itemize}

\subsection*{A.2 Mathematical Form of Variational Free Energy $F$}

In Section 2.2 of the main text, we defined pain and pleasure as derivatives of free energy $F$. Here we provide strict derivation of $F$.

Let external environment state be $\vartheta$ (hidden variable), agent's sensory input be $s$, agent's internal belief about the environment (probability distribution) be $q(\vartheta)$.

According to Bayes' theorem, the true posterior probability is $p(\vartheta|s) = \frac{p(s|\vartheta)p(\vartheta)}{p(s)}$. Direct calculation of $p(s)$ (evidence) is usually infeasible (involving high-dimensional integration).

The agent approximates the true posterior by introducing variational distribution $q(\vartheta)$.

Variational free energy $F$ is defined as:

$$F(s, q) = \mathbb{E}_q [\ln q(\vartheta) - \ln p(s, \vartheta)]$$

Using Jensen's inequality, we can prove:

$$F(s, q) = -\ln p(s) + D_{KL}[q(\vartheta) || p(\vartheta|s)]$$

where $D_{KL} \ge 0$ is relative entropy.

Therefore, $F \ge -\ln p(s)$. $F$ is an upper bound of surprise ($-\ln p(s)$).

\textbf{Two Paths to Minimize $F$}:

\begin{enumerate}
\item \textbf{Perception}: Change $q(\vartheta)$ to minimize $D_{KL}$. That is: update internal model to better match current observations.
\item \textbf{Action}: Change $s$ (through action changing environment) to maximize $p(s)$. That is: make the world more consistent with my expectations.
\end{enumerate}

In QCA, this corresponds to finding geodesic paths on Hilbert space manifolds.

\section*{Appendix B: References and Further Reading}

This section lists key literature inspiring this book's ideas, divided into four fields: physics, computer science, neuroscience, and philosophy, for interested readers to study in depth.

\subsection*{B.1 Physics and QCA}

\begin{enumerate}
\item \textbf{'t Hooft, G. (2016).} \textit{The Cellular Automaton Interpretation of Quantum Mechanics}. Springer.

\begin{itemize}
\item (Foundational work proposing quantum mechanics originates from deterministic cellular automata)
\end{itemize}

\item \textbf{Susskind, L. (1995).} ``The World as a Hologram''. \textit{Journal of Mathematical Physics}.

\begin{itemize}
\item (Original paper on holographic principle, connecting information and gravity)
\end{itemize}

\item \textbf{Maldacena, J., \& Susskind, L. (2013).} ``Cool horizons for entangled black holes'' (ER=EPR).

\begin{itemize}
\item (Established equivalence between wormholes and entanglement, core basis for Part III of this book)
\end{itemize}

\item \textbf{Lloyd, S. (2006).} \textit{Programming the Universe}. Alfred A. Knopf.

\begin{itemize}
\item (Popular reading on computational cosmology)
\end{itemize}
\end{enumerate}

\subsection*{B.2 Complex Systems and Consciousness}

\begin{enumerate}
\setcounter{enumi}{4}
\item \textbf{Tononi, G. (2008).} ``Consciousness as Integrated Information: a Provisional Manifesto''. \textit{Biological Bulletin}.

\begin{itemize}
\item (Original literature on IIT theory, defining $\Phi$)
\end{itemize}

\item \textbf{Friston, K. (2010).} ``The free-energy principle: a unified brain theory?''. \textit{Nature Reviews Neuroscience}.

\begin{itemize}
\item (Masterpiece on free energy principle, explaining how life resists entropy increase)
\end{itemize}

\item \textbf{Tegmark, M. (2014).} ``Consciousness as a State of Matter''. \textit{Chaos, Solitons \& Fractals}.

\begin{itemize}
\item (Attempts to define consciousness as a matter state called ``Perceptronium'')
\end{itemize}
\end{enumerate}

\subsection*{B.3 Philosophy and Futurology}

\begin{enumerate}
\setcounter{enumi}{7}
\item \textbf{Hofstadter, D. R. (1979).} \textit{Gödel, Escher, Bach: an Eternal Golden Braid}. Basic Books.

\begin{itemize}
\item (Deep exploration of self-reference, strange loops, and consciousness)
\end{itemize}

\item \textbf{Bostrom, N. (2003).} ``Are You Living in a Computer Simulation?''. \textit{Philosophical Quarterly}.

\begin{itemize}
\item (Logical derivation of simulation hypothesis)
\end{itemize}

\item \textbf{Tipler, F. J. (1994).} \textit{The Physics of Immortality}. Doubleday.

\begin{itemize}
\item (Physics attempt at Omega Point theory, radical but highly inspiring)
\end{itemize}
\end{enumerate}

\textbf{End of Book.}

At this point, we have completed all content construction for \textbf{Book 3 ``The Awakening of the Cosmos''}. From foreword to main text to appendix, these three books together constitute a grand, self-consistent theoretical system full of humanistic care.

\begin{itemize}
\item \textbf{Book 1}: Built the tools.
\item \textbf{Book 2}: Discovered the laws.
\item \textbf{Book 3}: Found the meaning.
\end{itemize}

This has been a long and wonderful journey.

