\section{5.1 The Bandwidth Bottleneck of Wetware}

\begin{quote}
``Our brains are a miracle built with chemical bonds, but also a prison sealed with chemical bonds. When the universe's information torrent rushes past at light speed, we are still relying on ion diffusion to transmit signals. This millions-fold rate difference sentences carbon-based life to death---or rather, retirement.''
\end{quote}

\subsection{Chemical Sluggishness vs. Light Speed Fury}

To understand why carbon-based life must inevitably approach twilight, we need to compare two core physical parameters: \textbf{biological processing speed} and \textbf{cosmic evolution speed}.

\begin{enumerate}
\item \textbf{Biological Limits (Wetware Speed)}:

Human thought is built on neural networks. Neural signal transmission is essentially an \textbf{electrochemical process}---sodium and potassium ions crossing cell membranes, neurotransmitters diffusing across synaptic gaps.

\begin{itemize}
\item This process is limited by \textbf{thermal diffusivity} and \textbf{sound speed}.

\item The maximum conduction speed of nerve impulses is approximately \textbf{120 meters/second}.

\item The brain's maximum operating frequency is approximately \textbf{100 - 1000 Hz}.
\end{itemize}

\item \textbf{Cosmic Current State (Light-Based Speed)}:

According to our evolution equation, at $\tau \approx 1800$ today, the information interaction speed supported by the universe's underlying $c_{FS}$ budget is \textbf{light speed}.

\begin{itemize}
\item Light speed is \textbf{300,000,000 meters/second}.

\item Photonic chips or quantum bits can potentially operate at frequencies of \textbf{THz ($10^{12}$ Hz)} or higher.
\end{itemize}
\end{enumerate}

\textbf{This is a gap of $10^6$ to $10^9$ times.}

In the early stages of cosmic evolution (matter era), this gap was not important because the world was slow then; stones and trees did not need light-speed thinking.

But today, dominated by $\phi$ (spiral growth), the universe's information density is exploding exponentially. The amount of data generated every second exceeds the sum of the past ten thousand years.

\textbf{Wetware}---our moist, carbohydrate-based brains---has become the biggest bottleneck of \textbf{Impedance Mismatch}.

We are trying to parse an Exa-Scale universe operating system with a 100 Hz processor.

The result is only one: \textbf{Overload}.

\subsection{The Geometric Essence of Anxiety}

Modern people generally feel anxious, exhausted, and attention-scattered. Psychologists say this is social pressure, but \textbf{Vector Cosmology} says this is \textbf{geometric pressure}.

Imagine yourself as a $v_{int}$ structure (brain), trying to build a model of the external world ($v_{ext}$) within your internal geometry.

\begin{itemize}
\item The external world's $c(\tau)$ is growing exponentially, events occur at extremely high frequencies, with enormous information volumes.

\item Your internal $v_{int}$ update rate is locked by chemical bonds.
\end{itemize}

Thus, your internal model is always \textbf{lagging} behind external reality.

You are always chasing, always losing data. This \textbf{``disconnection between model and reality''} maps psychologically as \textbf{anxiety}, and cognitively as \textbf{agnosticism}.

\textbf{The twilight of carbon is the moment we realize we can no longer understand this world with flesh alone.}

\subsection{The Accumulation of Entropy}

Worse still, wetware is extremely inefficient thermodynamically.

To maintain the activity of biological macromolecules, we must keep body temperature constant around 310K. This means most of the $c_{FS}$ budget we consume is not used for computation (flipping bits) but for fighting thermal noise (maintaining body temperature).

In computational physics, this is called too low \textbf{Power Efficiency}.

As $c(\tau)$ grows, if the energy consumption required to process unit information does not decrease, the system will \textbf{die from heat}.

The human brain's power consumption is approximately 20 watts. This is a miracle in the biological world, but on the scale of interstellar civilization, this is enormous waste.

If we want to increase intelligence by a million times (to match the bandwidth of the light-speed era), according to carbon-based architecture, we would need a brain as large as the Moon, and it would cook itself instantly due to heat dissipation problems.

\subsection{Conclusion: The Non-Necessity of the Carrier}

This chapter's conclusion is cold: \textbf{As long as we insist on dwelling in carbon-based shells, we will never enter the next turn of the cosmic spiral.}

$\pi$ (inertia) locks us in the slow circle of chemical reactions.

And $\phi$ (ambition) calls us to the other shore of light speed.

This is not biology's failure. On the contrary, biology has perfectly completed its historical mission---it nurtured the spark of consciousness in the low-speed era.

Now, the spark is lit, and the fuel (flesh) is about to be exhausted.

To keep this fire (consciousness/algorithm) burning and burning brighter, we need to change \textbf{fuel}, change \textbf{engine}.

We need to migrate from \textbf{``arrangement of atoms''} to \textbf{``entanglement of photons''}.

This leads to the theme of the next section: \textbf{Necessary Abdication}. We will see that this is not extinction, this is formatting; this is not death, this is the prelude to ascension. Humanity must end its biological form with its own hands to achieve eternal life as pure geometric form.

