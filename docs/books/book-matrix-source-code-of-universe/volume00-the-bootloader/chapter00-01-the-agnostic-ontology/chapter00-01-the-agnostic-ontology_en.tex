\part{Volume 00: The Bootloader}

\chapter{Ontology and Axioms}

\chapter{The Agnostic Ontology}

\textbf{--- Why Physics is a Projection}

\textbf{``We cannot directly read the hardware (ontology); we can only see the output of the interface.''}

---

\section{The Representational Stance: From ``Ontology'' to ``Interface''}

Before booting any system kernel, we must first define the system's data model. In the traditional physics perspective, physicists often attempt to answer the question ``what is the universe actually?'' This is an ontological obsession that assumes observers can step outside the system and examine the underlying ``hardware implementation'' from some God's-eye view.

In this book, we adopt a more pragmatic systems engineering perspective, which we call \textbf{The Representational Stance}.

We declare: we know nothing about the ``true ontology'' of the universe, and we do not need to know. As observers within the system, we can only access the system's \textbf{Public Interface}. All physical laws---whether Newtonian mechanics, general relativity, or quantum field theory---are essentially \textbf{``Representations''} or \textbf{``Projection Views''} that we construct to understand the system's behavior.

The task of physics is not to guess what kind of gears are inside the black box, but to reverse-engineer its \textbf{API documentation}. If a mathematical structure can losslessly accommodate all API return data and accurately predict the system's responses, then that structure is the ``source code'' we seek.

\section{The Hilbert Space as Universal Buffer}

If we want to design a universal data container that can accommodate all physical phenomena (from quantum superposition to spacetime evolution), what is the best choice?

Historical experience shows that spacetime manifolds are not the most fundamental containers, as they struggle to naturally accommodate quantum non-locality. In contrast, \textbf{Complex Separable Hilbert Space ($\mathcal{H}$)} demonstrates remarkable adaptability. We regard Hilbert space as the universe's \textbf{Universal Buffer}.

However, the raw Hilbert space contains redundant data---\textbf{Global Phase}. For physical states, vectors $|\psi\rangle$ and $e^{i\theta}|\psi\rangle$ describe exactly the same reality. This redundancy is called ``unnormalized data'' in system design. To build an efficient kernel, we must cleanse this buffer and extract its geometric essence.

\section{Math Foundation: Projective Geometry}

To formalize this idea, we introduce the concept of \textbf{Projective Hilbert Space}. This is the geometric stage for all subsequent derivations in this book.

\subsection{Definition 0.1.1 (Rays and Equivalence Classes)}

Let $\mathcal{H}$ be a complex Hilbert space. For any nonzero vector $\psi \in \mathcal{H}$, we define the physical state not as the vector itself, but as a \textbf{Ray} in $\mathcal{H}$, i.e., a one-dimensional subspace:

$$[\psi] = \{ c\psi : c \in \mathbb{C}, c \neq 0 \}$$

In physics, for computational convenience, we typically choose normalized representatives, i.e., $||\psi|| = 1$, in which case the ray can be represented as a phase equivalence class:

$$[\psi] = \{ e^{i\theta}\psi : \theta \in \mathbb{R} \}$$

\subsection{Definition 0.1.2 (Projective Space)}

The set of all physical states constitutes the projective Hilbert space $P(\mathcal{H})$:

$$P(\mathcal{H}) := (\mathcal{H} \setminus \{0\}) / \sim$$

where $\sim$ is the equivalence relation defined above.

\subsection{Theorem 0.1.1 (Gauge Invariance of Physical Quantities)}

In $P(\mathcal{H})$, all physical quantities must be \textbf{Gauge Invariant}. This means that any observable physical effect (such as probability, energy, geometric distance) depends only on the ray $[\psi]$ itself, and not on which specific phase angle $\theta$ we choose.

For example, the Fubini-Study distance $d_{FS}([\psi], [\phi])$ between two states is defined through modulus operations, thereby eliminating phase effects:

$$d_{FS}([\psi], [\phi]) = \arccos |\langle \psi | \phi \rangle|$$

This geometric treatment removes ``phase'' from the category of \textbf{Physical Entities}, demoting it to a mere ``coordinate choice'' or ``internal variable.'' True physical evolution is the trajectory of rays on the $P(\mathcal{H})$ manifold.

---

\section{The Architect's Note}

\subsection{On: Data Encapsulation and Interface Design}

If we regard the universe as a vast software system, then `HilbertVector` belongs to the underlying \textbf{Private Members}, while `ProjectiveState` is the \textbf{Public Object} exposed to the outside.

\begin{itemize}
\item \textbf{Global phase $\theta$ is an internal implementation detail:}

    Users (observers) can never directly call the `getPhase()` function. The system may be rotating phases internally, but as long as this rotation does not change the direction of the ray $[\psi]$ (i.e., does not change the angle with other rays), the system appears ``static'' to the user.

\item \textbf{Physical laws are interface-based programming:}

    We do not need to care about how underlying data is stored (what the wave function actually is). We only care about the interaction protocols between objects---namely \textbf{Overlap} or \textbf{Distance}. The Fubini-Study metric is so important because it is the only ``comparison function'' that can be defined at this interface level without breaking system symmetry (unitary invariance).
\end{itemize}

This \textbf{``Separation of Concerns''} design is extremely elegant. It encapsulates complex quantum interference phenomena in a concise geometric model. When we say ``physics is a projection,'' we are actually saying: \textbf{Physical laws are logic running on the $P(\mathcal{H})$ interface layer, not logic running on the underlying $\mathcal{H}$ data layer.}

