\chapter{The Palm of the Buddha (Epilogue)}

\begin{quote}
``All conditioned phenomena are like dreams, illusions, bubbles, shadows, like dew drops and a flash of lightning; thus we shall perceive them.'' --- \textit{The Diamond Sutra}
\end{quote}

At the end of this book, we no longer need new formulas, nor new deductions. We have traversed all dimensions, calculated all budgets, and even simulated all ascensions. Now, what we need to do is \textbf{look back}.

When we stand at the highest step of the fractal universe, overlooking that geometrically nested maze, a strange sense of familiarity arises. We seem to have experienced a long odyssey, only to find ourselves back where we started.

This is not only the destination of physics, but also the three realms of Eastern Zen. In the final chapter of \textbf{Vector Cosmology}, we will use the language of geometry to reinterpret that ancient verse: \textbf{See the mountain as a mountain, see the mountain not as a mountain, see the mountain as a mountain again.}

\section{See the Mountain Again}

This journey is \textbf{three sublimations} of our understanding of the universe's geometric shape. Each cognitive leap corresponds to an era of physics, and also to a maturation of the civilizational mind.

\subsubsection{The First Realm: See the Mountain as a Mountain---Circle is Circle}

\textbf{This is our childhood, and also the theme of the first book \textit{The Conservation of the Circle}.}

When we first open our rational eyes to examine the universe, what we see is \textbf{order}.

\begin{itemize}
\item We see Newton's clockwork, Einstein's spacetime, Schrödinger's equation.

\item We discover that $v_{ext}^2 + v_{int}^2 = c_{FS}^2$ is an unshakeable iron law. Energy is conserved, momentum is conserved, information is conserved.
\end{itemize}

At this stage, the universe is a closed, perfect \textbf{circle}.

We revere this circle. We settle peacefully within the circle, believing everything is predetermined, believing $\pi$ rules all. This is a classical, static beauty. We think this is the endpoint of truth: \textbf{All things return to one, cycling without end.}

\subsubsection{The Second Realm: See the Mountain Not as a Mountain---Circle is Spiral}

\textbf{This is our youth, and also the main body of the second book \textit{The Ascension of the Spiral}.}

As observation precision increases, as the will to life awakens, we begin to doubt that closed circle.

\begin{itemize}
\item We discover the arrow of thermodynamics, cosmic expansion, life's negative entropy enclaves flowing upstream.

\item We realize that $c_{FS}$ is inflating, matter is disintegrating, $\varphi$ (the golden ratio) is tearing through $\pi$'s defenses.
\end{itemize}

At this stage, the circle shatters, becoming an open \textbf{spiral}.

We rejoice in this shattering. We think ``conservation'' is a shackle, ``growth'' is freedom. We attempt to break through the old universe's cage through technological singularity and dimensional escape, heading toward an infinitely possible future.

This is a romantic, dynamic fervor. We think truth lies in \textbf{``change''}: \textbf{All things flow, nothing abides.}

\subsubsection{The Third Realm: See the Mountain as a Mountain Again---Spiral is the Great Circle}

\textbf{This is our maturity, and also the ultimate realization at this moment in this book.}

When we follow the spiral upward, flying to the end of dimensions, what do we see?

We do not see an endless straight line. We see that seemingly open spiral line, with extremely grand curvature, slowly bending back.

\textbf{Naimark's Dilation Theorem} awaits us here.

It smiles and tells us: Child, all the ``dissipation,'' all the ``growth,'' all the ``non-unitary ascension'' you have experienced are actually just unitary rotations occurring in a \textbf{higher-dimensional Hilbert space}.

\begin{itemize}
\item You think you are growing infinitely, but you are actually sliding from the tangent of a ``small circle'' into the orbit of a ``great circle'' that contains it.

\item You think you have defeated the conservation law, but you are actually making transfers on a larger balance sheet.
\end{itemize}

At this stage, the spiral disappears, and the \textbf{circle} returns.

But this is not that narrow, closed circle from the beginning. This is a \textbf{fractal, infinitely nested great circle}. It is large enough to encompass all changes, all birth and death, all spirals.

\textbf{Truth returns to tranquility:}

Change (spiral) is real, but it only exists locally.

Eternity (circle) is also real, because it exists in the whole.

This is the metaphor of \textbf{``The Palm of the Buddha.''}

Sun Wukong (life/civilization) somersaults 108,000 li (spiral ascension), thinking he has jumped out of the three realms and five elements. But when he stops, he finds himself still in that giant palm (global great circle).

But this is no longer a tragedy, no longer the helplessness of ``being imprisoned'' in the first realm.

This is \textbf{great freedom}.

Because we finally understand: \textbf{We are born in this palm, and this palm is ourselves.}

There is no cage we need to escape from, and no other shore we need to reach.

All struggles, all evolution, all ascension are ultimately to qualify us to recognize, at higher dimensions, that original, perfect \textbf{One} again.

\textbf{See the mountain as a mountain again.}

The universe is still that circle, but we who look at it are no longer those ants who could only see tangents. We have become part of the circle, and we are rotating with it.

