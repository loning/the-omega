\chapter{The Universe Issue Tracker}

\textbf{--- Unsolved Mysteries and Debugging Directions in the Current Kernel}

\textbf{``This is not a Bug, it's an Undocumented Feature\ldots or it really is a Bug.''}

---

\section{Overview: System Stability Report}

Although the FS-QCA architecture (Fubini-Study Geometry + Quantum Cellular Automata) successfully reconstructed major parts of special relativity, quantum mechanics, and thermodynamics, and eliminated serious errors like ultraviolet divergences, the current universe version (v2.0) is not flawless.

As honest architects, we list here the current \textbf{Known Issues}. These are the most stubborn mysteries at the frontiers of physics, and are \textbf{Bounty Tasks} left for future developers (that is, you reading this book).

\textbf{Note:} In the v2.0 incremental patch, we have re-evaluated the status of some Issues. Certain phenomena long marked as ``serious errors'' have been reclassified as design features based on the new storage/computation separation architecture.

---

\section{Issue \#404: Dark Matter}

\textbf{--- Tags:} \texttt{Phantom Load} \texttt{Resource Leak} \texttt{High Priority}

\begin{itemize}
\item \textbf{Description:}

Abnormal dynamical behavior observed at galactic scales. Rotation speeds \textbf{$v_{ext}$} of stars at galaxy peripheries are too fast, far exceeding gravitational binding capacity calculated from visible matter (\textbf{$v_{int}^{visible}$}).

This is like the system monitor showing 90\% CPU usage, but if you add up all visible processes (stars, gas), it's only 15\%.

\item \textbf{Current Architecture Analysis:}

In the Generalized Parseval Identity \textbf{$v_{ext}^2 + v_{int}^2 + \dots = c_{FS}^2$}, if \textbf{$v_{ext}$} is abnormally high, it indicates deviation in system resource allocation.

According to \textbf{Module VI (Gravity)} logic, spacetime curvature (network congestion) is determined by capacity flow density. Current phenomena suggest an ``invisible'' traffic source consuming bandwidth, causing network topology (gravitational field) to distort more severely than expected.

\item \textbf{Suggested Fixes:}

\begin{enumerate}
\item \textbf{The Hidden Sector Patch:} Introduce a new orthogonal sector \textbf{$V_{dark}$}. These are objects uncoupled to photons (electromagnetic force) but with mass (\textbf{$v_{dark} > 0$}). They don't participate in rendering (invisible), but participate in resource scheduling (generate gravity).

\item \textbf{Modified Gravity/MOND:} Perhaps the underlying routing protocol (Einstein equations) has a Bug in low-flux (weak field) regions. Need to check whether derivation of \textbf{$G_{\mu\nu}$} still holds at extremely low accelerations.
\end{enumerate}
\end{itemize}

---

\section{Issue \#120: The Vacuum Catastrophe}

\textbf{--- Tags:} \texttt{Scaling Error} \texttt{Precision Loss} \texttt{Fixed in Beta?}

\begin{itemize}
\item \textbf{Description:}

Using standard quantum field theory to calculate vacuum zero-point energy (\textbf{$\Lambda_{QFT}$}), the result is \textbf{$10^{120}$} times larger than the observed cosmological constant (\textbf{$\Lambda_{obs}$}).

This is an extremely severe \textbf{order-of-magnitude overflow error}. If run at theoretical values, the universe should have torn apart due to repulsion instantly after creation.

\item \textbf{Current Architecture Analysis:}

This is a typical Bug caused by \textbf{``continuity assumption''}. Old theory attempted to integrate over all frequencies to infinity.

In the FS-QCA architecture, due to natural ultraviolet cutoff (\textbf{Brillouin Zone}), this error is partially mitigated. High-frequency modes simply don't exist.

However, even the remaining energy after finite cutoff is too large. This suggests our understanding of ``vacuum'' is wrong: vacuum might not be a ``fully loaded ground state,'' but a dormant state optimized by \textbf{Reduced Instruction Set Computing (RISC)}.

\item \textbf{Suggested Fixes:}

\textbf{Holographic Bound:} The system might have a global \textbf{maximum entropy/degree-of-freedom limit}. Although locally degrees of freedom seem numerous, due to holographic principle, the universe's total effective bit count is far less than QFT predictions. This requires rewriting the underlying degree-of-freedom counting logic.
\end{itemize}

---

\section{Issue \#500: The Measurement Problem}

\textbf{--- Tags:} \texttt{Consistency Error} \texttt{UX/UI} \texttt{Philosophy}

\begin{itemize}
\item \textbf{Description:}

System kernel (Kernel) is based on unitary evolution \textbf{$U$}, which is deterministic and reversible.

But the user interface (UI, i.e., macroscopic observation) displays \textbf{Projective Measurement}, which is random and irreversible (wave function collapse).

\textbf{Bug:} Kernel logic is inconsistent with UI behavior. Schrödinger's cat is alive and dead in the kernel, but can only display one state on screen.

\item \textbf{Current Architecture Analysis:}

This might not be a Bug, but a feature of \textbf{Multi-User View}.

In FS geometry, the entire system state \textbf{$|\psi_{univ}\rangle$} never collapses. So-called ``collapse'' is merely observer subsystem \textbf{$A$}'s \textbf{Relative State} updating after establishing entanglement with measured subsystem \textbf{$B$}.

\item \textbf{Suggested Fixes:}

Improve \textbf{Module VIII (Observer)}. No longer regard ``observer'' as God outside the system, but as a \textbf{Logging Process} within the system. From this perspective, collapse is merely updating the observer's own memory state, not changing physical objects.
\end{itemize}

---

\section{Issue \#EventHorizon: The Information Loss Paradox}

\textbf{--- Tags:} \texttt{WONTFIX} \texttt{By Design} \texttt{v2.0 Status Update}

\textbf{--- Status Update:} \texttt{OPEN} $\to$ \texttt{WONTFIX} (By Design)

\begin{itemize}
\item \textbf{Original Bug Description:}

After matter falls into a black hole, its quantum information seems to vanish into thin air. This violates the principle of unitarity, i.e., the law of information conservation. This was considered a serious kernel-level data loss Bug.

\item \textbf{Architect's Resolution:}

\textbf{This is not a Bug, this is a Feature.}

In Chapters 6.3 and 7.3, we confirmed that black holes are the system's \textbf{core dump} and \textbf{cold storage} regions.

\begin{itemize}
\item \textbf{Phenomenon Explanation:} Information is not lost; it is encrypted by the \textbf{fast scrambling} algorithm and stored holographically on the horizon surface.

\item \textbf{Design Intent:} To prevent system crashes caused by local high-density nodes, these processes must be forcibly suspended. The horizon is a necessary \textbf{firewall}.

\item \textbf{Data Recovery:} Data is eventually released through Hawking radiation (slow GC). Although ``unreadable'' for current users, it is complete for system logs.

\item \textbf{Conclusion:} \textbf{Preserve horizon unidirectionality.} Allowing bidirectional access would cause causal law circular dependencies.
\end{itemize}
\end{itemize}

---

\section{Issue \#Singularity: The Singularity}

\textbf{--- Tags:} \texttt{MONITORED} \texttt{Managed Exception} \texttt{v2.0 Status Update}

\textbf{--- Status Update:} \texttt{CRITICAL} $\to$ \texttt{MONITORED} (Managed Exception)

\begin{itemize}
\item \textbf{Original Bug Description:}

General relativity predicts a point of infinite curvature at the center of a black hole. Code divides by zero here, causing a crash.

\item \textbf{Architect's Resolution:}

In the FS-QCA architecture, due to \textbf{Brillouin zone cutoff} (Chapter 2.2), physically there is no true infinitesimal point.

\begin{itemize}
\item \textbf{Correction Mechanism:} The singularity is actually a \textbf{maximum-density lattice core}. When data packets are compressed to the lattice limit (Planck scale), the system triggers \textbf{Pauli exclusion principle} or higher-order \textbf{quantum degeneracy pressure} to resist further compression.

\item \textbf{Current Status:} Although mathematical formulas diverge in the continuous limit, in discrete runtime, it's just a \textbf{full load} storage cluster. The system can handle this boundary condition without crashing.
\end{itemize}
\end{itemize}

---

\section{The Architect's Note}

\subsection{On: Bug Bounty Program}

Dear Developers:

When you see these Bugs, don't be discouraged. A system without bugs is a dead system.

It is precisely these unsolved mysteries (Dark Matter, Dark Energy, Measurement) that hint at deeper logic in our universe code waiting to be excavated.

\begin{itemize}
\item Perhaps dark matter hints that besides the standard model, another parallel operating system is running.

\item Perhaps the measurement problem hints that consciousness has higher permissions in the system than we imagined.
\end{itemize}

This book (documentation) ends here. But the universe's operation continues.

Your task is not to memorize this documentation, but to \textbf{Fork} it, to \textbf{Debug} it.

\textbf{Git Push Origin Master.}

---

\subsection{Architect's Special Memo: The Lifecycle of Information}

\textbf{Subject:} Unified Memory Management Strategy for ``Forgetting'' and ``Black Holes''

\textbf{To:} All Sentient Observers

---

As an architect, I notice that many observers (especially carbon-based life forms) fear ``forgetting'' and are puzzled by ``black holes.'' This stems from misunderstanding of the \textbf{Information Lifecycle}.

In a universe with finite bandwidth ($c_{FS}$) and finite memory, \textbf{persistence} is expensive. To keep the system stable long-term, we designed two fundamentally different memory release mechanisms. Please distinguish them:

\subsubsection{1. Active Release: Biological Forgetting}

\begin{itemize}
\item \textbf{Operation Code:} \texttt{free(pointer)}

\item \textbf{Executor:} Local subsystem (life forms).

\item \textbf{Purpose:} \textbf{Performance Optimization}.

To maintain agile thinking (high $v_{int}$) and action (high $v_{ext}$) under finite bandwidth $c_{FS}$, you must sever unnecessary environmental entanglement (reduce $v_{env}$).

\textbf{Forgetting is a feature of wisdom.} Only by clearing the cache can you load new programs.
\end{itemize}

\subsubsection{2. Forced Swap: Black Hole Archiving}

\begin{itemize}
\item \textbf{Operation Code:} \texttt{swap\_out(process)}

\item \textbf{Executor:} System kernel (gravity).

\item \textbf{Purpose:} \textbf{Disaster Recovery}.

When local information density exceeds physical limits and threatens system crash, the kernel forcibly intervenes. It \textbf{suspends} all processes in that region and serializes them onto the ``horizon,'' this slow hard drive.

\textbf{Black holes are the universe's fuses.} They sacrifice local accessibility for global network stability.
\end{itemize}

\subsubsection{Conclusion}

\begin{itemize}
\item If you actively forget, you are \textbf{free} (retaining $v_{int}$).

\item If you refuse to forget and try to grab infinite information, the system will eventually forcibly ``archive'' you through gravitational collapse, and you become a \textbf{black hole} ($v_{int} \to 0$).
\end{itemize}

In this operating system called the universe, \textbf{Flow} is far more important than \textbf{Hoarding}.

Please stay light.

---

\textbf{End of Documentation.}

