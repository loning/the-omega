\chapter{Gravity: Market Distortion}

\begin{quote}
``Matter tells spacetime how to allocate budget; spacetime tells matter how to move according to the remaining budget.''
\end{quote}

In the previous two chapters, we established a universe model based on a single-particle perspective: a particle is like a lone trader, engaging in a zero-sum game between its own $v_{ext}$ (motion) and $v_{int}$ (existence).

However, the universe is not alone. When countless particles gather together---forming rocks, planets, stars, and even galaxies---quantitative change triggers qualitative change. These massive \textbf{$v_{int}$ aggregates} are no longer merely participants in the market; they become rule-makers.

In general relativity, this is called ``spacetime curvature.'' But from the economic perspective of \textbf{Vector Cosmology}, this should be more precisely described as a \textbf{``Market Monopoly''}.

\section{The Monopolist}

If photons are free wanderers, then massive celestial bodies like Earth or the Sun are the universe's top \textbf{oligopolistic monopolists}.

\subsubsection{High-Density Asset Accumulation}

Let us recall the essence of mass: mass is \textbf{frozen $c_{FS}$ budget}. A proton with enormous $v_{int}$ means it consumes an extremely high information update rate in the microscopic dimension.

Now, imagine compressing $10^{57}$ protons (the mass of the Sun) into a relatively small spatial region. This creates a terrible consequence: \textbf{regional computational power depletion}.

In this vast Quantum Cellular Automata (QCA) network that is the universe, each spatial node (or Planck volume) has an upper limit on the FS capacity density it can carry. When a massive object occupies this region, it is essentially declaring: ``All bandwidth in this region is used by me to maintain my internal existence ($v_{int}$).''

This is \textbf{high-density asset accumulation}. The location of the Sun is no longer a flat stage but a massive \textbf{budget black hole} (here ``black hole'' is a metaphor, referring to resource consumption). It monopolizes local $c_{FS}$, causing severe distortion of the ``background capacity'' in that region.

\subsubsection{The Crowding-Out Effect}

What does this monopoly mean for surrounding space? It creates a \textbf{crowding-out effect}.

In the deep vacuum far from mass, spatial nodes are idle, and any passing photon or particle can freely apply for $c_{FS}$ budget for $v_{ext}$ (displacement). The ``transaction cost'' there is low, and the geometric structure is flat.

But when you approach a massive object (monopolist), the situation changes. Since the monopolist has already requisitioned most of the underlying degrees of freedom to encode its own mass information, the effective budget available to ``passers-by'' becomes scarce.

This scarcity manifests geometrically as changes in the \textbf{metric}.

In general relativity, we say that time flows slower in gravitational potential wells (gravitational redshift). In our vector language, this is because \textbf{the monopolist has seized the refresh rate}.

\begin{itemize}
\item On Earth's surface, the underlying network of space is busy processing Earth's own massive $v_{int}$ data stream.

\item When you stand on the ground, the local environment you are in has its ``background refresh rate'' actually ``dragged down'' by Earth's mass. To maintain total budget balance, any observer in this strong gravitational field has their available $c_{FS}$ share \textbf{shrunk} relative to distant observers.
\end{itemize}

You have not slowed down; it is the \textbf{``market''} you are in that has slowed down.

\subsubsection{From Curvature to Gradient}

Therefore, gravity is no longer a mysterious force emanating from large objects and pulling small objects. Gravity is a \textbf{gradient caused by uneven budget distribution}.

Imagine a huge spreadsheet representing the space of the universe.

\begin{itemize}
\item In vacuum regions, each cell's value (available budget) is 100.

\item At the Sun's location, the cell's value becomes 1 (because 99 are occupied by the Sun).

\item Between the two, a \textbf{descending gradient} from 100 to 1 is formed.
\end{itemize}

When a small asteroid (test particle) enters this region, it instinctively follows \textbf{Fermat's principle} or the \textbf{principle of least action}. It is not ``pulled'' toward the Sun; it is trying to find an optimal path in a market where \textbf{liquidity is gradually drying up}.

This explains why light bends when passing the Sun. Light does not feel a force; light is merely traversing a region of \textbf{``computational congestion''}. To maintain its constant $c_{FS}$ consumption (light speed), it must adjust its path to adapt to the distortion of local spatial geometry (budget density).

\textbf{Conclusion:} Gravity is the \textbf{management cost} the universe must pay to handle high-density information accumulation. Massive objects, by monopolizing $v_{int}$, distort the $v_{ext}$ trading rules around them. So-called ``universal gravitation'' is merely the \textbf{path correction} that free particles are forced to make when facing market monopoly.

