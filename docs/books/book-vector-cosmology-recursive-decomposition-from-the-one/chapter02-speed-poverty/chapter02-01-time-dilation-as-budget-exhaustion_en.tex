\chapter{The Poverty of Speed}

\begin{quote}
``Light did not transcend time; light lost time because of poverty.''
\end{quote}

We have established the macroscopic architecture of the universe: a constant information-velocity budget $c_{FS}$, and a mandatory Pythagorean allocation law. Now, we will use this entirely new perspective to re-examine the most famous and perplexing theory in modern physics---special relativity.

In textbooks, relativity is often described as some mysterious property of spacetime geometry: length contraction, time dilation, constant speed of light. But in \textbf{Vector Cosmology}, all of this is no longer an axiom; it is an inevitable corollary of ``budget conservation.''

\section{The Truth of Relativity: Time Dilation as Budget Exhaustion}

What do we really mean when we say ``time has slowed down''?

On a high-speed spacecraft, astronauts' heartbeats slow down, atomic vibrations slow down, and even random processes like decay slow down. Einstein told us this is because ``each second now is longer than before.'' This sounds extremely mysterious.

But under the \textbf{information-velocity decomposition} framework of Vector Cosmology, the truth is much more straightforward and harsh: \textbf{Your time slows down because you don't have enough budget to pay for the passage of time.}

\subsubsection{Internal Velocity as Time Flow Rate}

Recall our core formula:

\[v_{ext}^2 + v_{int}^2 = c_{FS}^2\]

We need to assign a physical meaning to $v_{int}$ (internal velocity). It represents the rate at which the vector evolves in the internal sector. For a physical system (such as a clock or a person), \textbf{``being alive''} or \textbf{``experiencing time''} is essentially the process of continuous internal state changes.

If $v_{int}$ is large, it means the system's internal evolution is intense, and time passes quickly; if $v_{int}$ goes to zero, it means the system's internal state is completely frozen, and time stops.

Therefore, \textbf{$v_{int}$ is your proper time flow rate}.

\subsubsection{A Zero-Sum Game}

Now, let's see what happens when you try to accelerate.

Suppose you are initially at rest in space. At this point, your external velocity $v_{ext} = 0$. According to the Pythagorean identity, all the budget $c_{FS}$ you possess is invested in internal evolution:

\[v_{int} = c_{FS}\]

At this moment, your time flows at its fastest rate; your life is burning at the maximum rate allowed by the universe.

When you start accelerating, trying to gain velocity $v$ in space (i.e., increasing $v_{ext}$). Because the total budget $c_{FS}$ is locked (the universe won't give you extra credit), you must divert part of the allocation from $v_{int}$ to pay for the cost of $v_{ext}$.

Your internal evolution is forced to slow down:

\[v_{int} = \sqrt{c_{FS}^2 - v^2}\]

This is the physical mechanism of \textbf{time dilation}. It is not magic of the spacetime background; it is \textbf{resource squeeze}. When you use more and more capacity for ``changing position,'' you are left with less and less capacity for ``changing yourself.''

\subsubsection{Geometric Derivation of the Lorentz Factor}

We can directly derive the core mathematical structure of special relativity---the Lorentz factor $\gamma$---from this simple circle equation.

In physics, we are accustomed to using $c$ to denote the speed of light (our $c_{FS}$), and $v$ to denote spatial velocity (our $v_{ext}$). Then the ratio of internal time flow rate $v_{int}$ to the maximum rate $c$ is:

\[\frac{v_{int}}{c} = \sqrt{1 - \frac{v^2}{c^2}}\]

If we denote the reference time flow at rest as $dt$, and the proper time flow in motion as $d\tau$, then by definition, $d\tau$ should be proportional to $v_{int}$. Thus we obtain:

\[\frac{d\tau}{dt} = \frac{v_{int}}{c} = \sqrt{1 - \frac{v^2}{c^2}}\]

Flipping this over gives the standard relativistic time dilation formula:

\[dt = \frac{d\tau}{\sqrt{1 - \frac{v^2}{c^2}}} = \gamma d\tau\]

See, we don't need to assume the constancy of the speed of light, nor do we need to introduce the complexity of Minkowski spacetime. We only need to acknowledge that the universe is a \textbf{circle}. The Lorentz factor $\gamma$ is actually the \textbf{secant function}---it is a simple trigonometric ratio of the hypotenuse to the leg in the geometry of the great circle division.

\subsubsection{The Poverty of Speed}

This perspective completely changes our view of ``extreme speed.''

In science fiction, the speed of light is portrayed as ultimate freedom. But in Vector Cosmology, the speed of light represents \textbf{ultimate poverty}.

When a particle's external velocity $v_{ext}$ reaches $c_{FS}$, the formula becomes:

\[c_{FS}^2 + v_{int}^2 = c_{FS}^2 \implies v_{int} = 0\]

This means its internal budget is completely depleted. It has no remaining capacity to undergo any internal changes. For a photon, from its birth in the Big Bang to its absorption by your retina, the thirteen billion years that span this interval are \textbf{zero seconds} in its subjective perspective. It does not age, does not evolve, because it has sacrificed all its ``existence'' to ``distance.''

\textbf{The truth of relativity is:} Motion is an expensive consumption. The reason we cannot exceed the speed of light is not because there is a wall ahead, but because when $v_{ext} = c_{FS}$, we are already bankrupt---we have no remaining budget left to convert into speed.

