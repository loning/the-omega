\chapter*{Glossary of Terms}

This book constructs a new physical language that integrates geometry, quantum information, and thermodynamics into a unified framework. To help readers better understand the core ideas of \textbf{Vector Cosmology}, the following lists key terms appearing in the book and their physical definitions.

\section*{Fundamental Concepts}

\begin{description}
\item[\textbf{Global Pure State Vector ($|\Psi\rangle$)}] The ontological foundation of the universe. A single, normalized quantum state vector inhabiting infinite-dimensional projective Hilbert space $P(\mathcal{H})$. All physical phenomena in the universe (particles, fields, spacetime) are projections of this vector onto different orthogonal subspaces.

\item[\textbf{Fubini-Study Capacity ($c_{FS}$)}] The universe's intrinsic change budget. It defines the maximum rate at which the global vector evolves in projective Hilbert space within intrinsic time $\tau$ ($||\dot{\Psi}||_{FS} = c_{FS}$). Microscopically, it corresponds to the information update frequency limit of quantum cellular automata; macroscopically, it determines the value of light speed $c$.

\item[\textbf{Intrinsic Time ($\tau$)}] The universe's ``master clock'' defined based on FS arc length. Unlike coordinate time that depends on reference frames, intrinsic time is a geometrically absolute parameter measuring the cumulative degree of objective change in the system's quantum state.

\item[\textbf{Pythagorean Identity}] The conservation law axiom governing the entire book: $v_{ext}^2 + v_{int}^2 + v_{env}^2 = c_{FS}^2$. It shows that the universe's total budget is constant, and physical evolution is essentially a zero-sum game of budget allocation among external motion, internal structure, and environmental dissipation.
\end{description}

\section*{Velocity Components \& Sectors}

\begin{description}
\item[\textbf{External Velocity ($v_{ext}$)}] The rate of change of the global vector's projection in the eigen-direction of ``position'' or ``spatial'' operators. Corresponds to spatial velocity or momentum in classical physics. When $v_{ext}$ reaches $c_{FS}$, the system moves at light speed.

\item[\textbf{Internal Velocity ($v_{int}$)}] The rate of change of the global vector's projection in the eigen-direction of ``internal degrees of freedom.'' Corresponds to intrinsic properties such as rest mass, spin, charge. It is the ``static budget'' required to maintain material existence.

\item[\textbf{Environmental Velocity ($v_{env}$)}] The rate of change of the global vector's projection in unobservable or uncontrollable degrees of freedom. It quantifies the growth rate of quantum entanglement and information dissipation, and is the geometric source of entropy increase and irreversibility.

\item[\textbf{The Dirac Circle}] The geometric representation of the relativistic dispersion relation $E^2 = p^2c^2 + m^2c^4$ on the information-velocity plane $(v_{ext}, v_{int})$. It reveals that mass is merely energy curled in internal dimensions.
\end{description}

\section*{Micro-Mechanism}

\begin{description}
\item[\textbf{Quantum Cellular Automaton (QCA)}] A microscopic discrete model of cosmic spacetime. A unitary evolution system driven by local interaction rules operating on a regular lattice. It provides natural ultraviolet cutoff for physics.

\item[\textbf{Lieb-Robinson Velocity ($v_{LR}$)}] The microscopic causal speed limit of information propagation in lattice systems. It is the physical origin of macroscopic light speed $c$, determined by lattice spacing and update time step.

\item[\textbf{Lattice Droop}] When particle momentum approaches the Brillouin zone boundary (Planck scale), due to discrete geometric effects, the perfect circular symmetry of relativity breaks, leading to the phenomenon $v_{ext}^2 + v_{int}^2 < c_{FS}^2$. This is the characteristic fingerprint of discrete spacetime.
\end{description}

\section*{Topology \& Information}

\begin{description}
\item[\textbf{Levinson's Knot}] A physical image based on Levinson's theorem. Material particles are viewed as topological structures formed by scattering phase winding $\pi$ angles in energy space. The existence of matter is essentially counting the circle constant $\pi$.

\item[\textbf{Wigner-Smith Time Delay}] The time particles linger in the interaction region during scattering. Geometrically, it equals the integral of FS velocity along the energy axis. Large delay means the vector has traversed enormous arc length in internal geometric dimensions.

\item[\textbf{Entropic Speed Limit}] The maximum rate of system entropy increase determined by $c_{FS}$: $|\dot{S}| \le c_{FS} \ln d$. It shows that chaos generation is limited by the universe's information processing bandwidth, thus ensuring the non-instantaneity of thermodynamic processes.

\item[\textbf{Negentropy Enclave}] The physical definition of living systems. A dissipative structure that actively suppresses its own $v_{env}$ and maintains high $v_{int}$ by consuming external low-entropy energy sources, locally achieving evolution against the thermodynamic arrow.
\end{description}

