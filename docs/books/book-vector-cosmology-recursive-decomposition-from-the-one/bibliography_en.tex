\chapter*{Bibliography}

The theoretical framework of this book is not built in the air but on solid research foundations in modern physics, geometry, and information theory. The following literature constitutes the academic foundation of \textbf{Vector Cosmology}, covering core ideas from scattering theory, geometric quantum mechanics to thermodynamic gravity and quantum cellular automata.

\section*{Fundamental Scattering Theory and Levinson's Theorem}

\begin{itemize}
\item[{[1]}] N. Levinson, \textit{On the determination of the potential from the phase of the scattering matrix}, Det. Kgl. Danske Videnskabernes Selskab. Matematisk-fysiske Meddelelser \textbf{25}, 1 (1949).

    \textit{Note: This is the core mathematical source of ``matter as counting,'' establishing the topological relationship between scattering phase and bound state number.}

\item[{[11]}] E. P. Wigner, \textit{Lower limit for the energy derivative of the scattering phase shift}, Physical Review \textbf{98}, 145 (1955).

    \textit{Note: Wigner laid the foundation for time delay here, linking phase derivative with time.}

\item[{[12]}] F. T. Smith, \textit{Lifetime matrix in collision theory}, Physical Review \textbf{118}, 349 (1960).

    \textit{Note: Further developed time delay operator theory, providing the operator foundation for our ``energy space geometry.''}

\item[{[13]}] R. G. Newton, \textit{Scattering theory of waves and particles} (Springer, 1982).

    \textit{Note: Classic textbook on scattering theory, providing detailed exposition of related mathematical structures.}
\end{itemize}

\section*{Geometric Quantum Mechanics and Fubini-Study Metric}

\begin{itemize}
\item[{[2]}] Y. Aharonov and J. Anandan, \textit{Phase change during a cyclic quantum evolution}, Physical Review Letters \textbf{58}, 1593 (1987).

    \textit{Note: Pioneering work introducing geometric phase into quantum evolution, laying the foundation for geometric description of projective Hilbert space.}

\item[{[3]}] J. Anandan and Y. Aharonov, \textit{Geometry of quantum evolution}, Physical Review Letters \textbf{65}, 1697 (1990).

    \textit{Note: Formally established Fubini-Study distance as the natural metric for quantum evolution.}

\item[{[5]}] E. R. Caianiello, \textit{Geometry from quantum mechanics}, Il Nuovo Cimento B \textbf{61}, 1 (1981).

    \textit{Note: Early attempt to derive physical spacetime from geometric structures of quantum mechanics.}
\end{itemize}

\section*{Quantum Speed Limits (QSL)}

\begin{itemize}
\item[{[4]}] N. Margolus and L. B. Levitin, \textit{The maximum speed of dynamical evolution}, Physica D: Nonlinear Phenomena \textbf{120}, 188 (1998).

    \textit{Note: Famous Margolus-Levitin bound, proving that system evolution speed is limited by average energy, a precursor to the ``budget limitation'' idea in this book.}

\item[{[6]}] S. Deffner and S. Campbell, \textit{Quantum speed limits: from heisenberg's uncertainty principle to optimal quantum control}, Journal of Physics A: Mathematical and Theoretical \textbf{50}, 453001 (2017).

    \textit{Note: Modern review on quantum speed limits, covering variance-based geometric bounds.}
\end{itemize}

\section*{Quantum Cellular Automata (QCA) and Lieb-Robinson Bounds}

\begin{itemize}
\item[{[8]}] E. H. Lieb and D. W. Robinson, \textit{The finite group velocity of quantum spin systems}, Communications in Mathematical Physics \textbf{28}, 251 (1972).

    \textit{Note: Proved the existence of maximum signal speed (light cone) in non-relativistic lattice systems, mathematical proof of the origin of light speed in this book's microscopic engine (QCA).}

\item[{[9]}] B. Schumacher and R. F. Werner, \textit{Reversible quantum cellular automata}, arXiv preprint quant-ph/0405174 (2004).

    \textit{Note: Established fundamental theory of reversible QCA, providing a model for unitary discrete evolution of the universe.}

\item[{[10]}] P. Arrighi, V. Nesme, and R. Werner, \textit{Unitarity of quantum cellular automata}, Journal of Computer and System Sciences \textbf{77}, 728 (2011).

    \textit{Note: Explored unitary conditions of QCA, ensuring microscopic information conservation.}
\end{itemize}

\section*{Thermodynamic Gravity and Holographic Principle}

\begin{itemize}
\item[{[7]}] T. Jacobson, \textit{Thermodynamics of spacetime: the einstein equation of state}, Physical Review Letters \textbf{75}, 1260 (1995).

    \textit{Note: Revolutionarily proposed that Einstein field equations are thermodynamic equations of state, inspiring the ``gravity as entropic force'' perspective in this book.}

\item[{[14]}] T. Jacobson, \textit{Entanglement equilibrium and the einstein equation}, Physical Review Letters \textbf{116}, 201101 (2016).

    \textit{Note: Further linked entanglement entropy with gravitational dynamics, supporting the ``geometry as entanglement'' picture.}

\item[{[15]}] E. Verlinde, \textit{On the origin of gravity and the laws of newton}, Journal of High Energy Physics \textbf{2011}, 1 (2011).

    \textit{Note: Proposed a complete account of gravity as entropic force, challenging gravity's status as a fundamental force.}
\end{itemize}

