\section{The Standing Wave of Now}

\begin{quote}
``The past has vanished into smoke, the future has not yet collapsed into form. The only territory we possess on the time axis is that fleeting `now.' But this is not a geometric point; it is an island built of memory and expectation, never sinking.''
\end{quote}

In classical physics, ``now'' is a concept that does not exist.

In Newton's or Einstein's equations, time $t$ is merely a coordinate axis. On this axis, past, present, and future are equal. Physical laws do not distinguish which moment is ``now,'' just as a map does not distinguish which coordinate is ``here.'' ``Now'' is merely a cursor moving with the observer.

But in the consciousness geometry of \textbf{Vector Cosmology}, ``now'' possesses a completely different physical substantiality. It is no longer a slice without thickness; it is a \textbf{structure with thickness}.

It is a \textbf{Standing Wave} that consciousness, this recursive strange loop, creates in the river of time.

\subsubsection{The Specious Present: Psychological Time}

Why can we hear melodies?

If time were instantaneous slices, then at moment $t$, we should only hear an isolated note. The previous note has already disappeared, the next note has not yet appeared. But our consciousness can magically ``glue'' together a series of discrete notes, perceiving the flow of \textbf{melody}.

This shows that our consciousness does not live at the physical time point $t$, but in a \textbf{time window $\Delta t$}. Psychologist William James called this \textbf{``The Specious Present''}.

From the perspective of FS geometry, this originates from the \textbf{``simulation of simulation''} mentioned in the previous section.

\begin{itemize}
\item When the brain processes information, recursive loops cause \textbf{delay}.

\item Input at moment $t$ does not immediately flow away; it is copied, cycled, and re-input into computations at moment $t+1$.

\item Thus, in consciousness's internal geometry, information from moments $t$, $t-1$, $t-2$ is \textbf{superimposed} together.
\end{itemize}

This is the \textbf{thickness of ``now''}. Like a long-exposure photograph, consciousness compresses all light and shadow from a period of time onto the same negative. The ``present'' we feel is actually the \textbf{reverberation} of the past few hundred milliseconds in Hilbert space.

\subsubsection{The Dynamics of Standing Waves}

This reverberation mathematically constitutes a \textbf{standing wave}.

Imagine a river (the passage of intrinsic time $\tau$). If you insert a stake (matter) into the river, water will flow around it. But if you create a whirlpool (self-referential consciousness) in the river, water will be drawn into it, spinning there.

Although the water molecules (QCA update steps) composing the whirlpool are updated every moment, the \textbf{shape of the whirlpool} remains unchanged.

Consciousness is this whirlpool on the time axis.

\begin{itemize}
\item \textbf{$v_{ext}$ flows away}: External world events occur and then vanish.

\item \textbf{$v_{int}$ remains}: Internal recursive computation, by consuming massive $c_{FS}$ budget, forcibly ``grabs'' information that should have dissipated into $v_{env}$, making them spin a few more rounds in the strange loop.
\end{itemize}

It is precisely this \textbf{``upstream grabbing''} that creates our sense of still living in ``now.'' Without this mechanism, we would be like photons: though experiencing billions of years, having no experience of time passage (photons have $v_{int}=0$, no standing wave).

\subsubsection{Consciousness is a Time Machine}

In this sense, every conscious brain is a miniature \textbf{time machine}.

It cannot send the body back to the past, but it can ``smuggle'' past information into the future.

When you recall a childhood scene, you are actually, within your brain (the $v_{int}$ sector), using current $c_{FS}$ budget to \textbf{reconstruct} a wave function projection from decades ago.

This is an expensive geometric operation. Maintaining the amplitude of the ``now'' standing wave requires consuming extremely high energy (glucose/oxygen). Once energy supply is interrupted (death), the recursive loop breaks, and the standing wave collapses.

At that moment, ``now'' disappears, and time returns to that cold, linear physical coordinate axis.

\subsubsection{Conclusion: The Surfer}

So, we are not passively drifting with the current.

We are \textbf{surfers}.

In the massive temporal torrent pushed by thermodynamics' arrow, consciousness, through self-referential strange loops, stands on the wave's crest, maintaining dynamic balance.

The surfboard beneath our feet is that \textbf{``thick now''} woven by recursive algorithms.

As long as we are still thinking, as long as the strange loop is still rotating, we will never fall into the abyss of ``past'' nor be swallowed by the mist of ``future.'' We always stand in \textbf{now}.

Since we already stand on time's wave crest, since we already possess ``now'' through self-reference, can observers use this special geometric position to influence or even determine that undetermined ocean?

This leads to the theme of the next chapter: \textbf{The Privilege of the Observer}. We will see that when we gaze into the abyss, we are not merely looking; we are \textbf{legislating}.

