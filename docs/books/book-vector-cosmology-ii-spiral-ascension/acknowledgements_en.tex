\chapter*{Acknowledgements: Shoulders of Giants}

The birth of this book is not the product of my personal fabrication, but a reweaving of the wisdom accumulated by human civilization over thousands of years in physics, mathematics, and philosophy.

The grand vision of \textbf{Vector Cosmology} is built upon the shoulders of a group of intellectual giants. Here, I pay my deepest respects to these pioneers.

\section{The Foundation of Mathematics and Geometry}

First, I thank \textbf{Guido Fubini} and \textbf{Eduard Study}. It was their definition of the \textbf{Fubini-Study metric} a century ago that provided the most fundamental geometric stage for this book. Without this metric, we could not quantify the cost of ``change,'' nor derive that Pythagorean identity that governs everything.

I thank \textbf{Norman Levinson}. His \textbf{Levinson's theorem} is the mathematical soul of this book's view that ``matter is counting.'' He showed us how topology transforms ethereal phases into indestructible particles.

I thank \textbf{Mark Naimark}. His \textbf{dilation theorem} provided the ultimate geometric redemption for the second book, allowing us to rigorously prove mathematically that ``spiral is the great circle,'' thereby reconciling the contradiction between evolution and conservation.

\section{Beacons of Physics}

In the field of physics, my gratitude is beyond words:

\begin{itemize}
\item \textbf{Paul Dirac}: His equation predicted antimatter, and his name is used by us to name that geometric circle connecting mass and energy (the \textbf{Dirac circle}).

\item \textbf{John Wheeler}: His profound insights about the ``participatory universe'' and ``it from bit'' are the spiritual guidance for this book's chapters on consciousness and observation.

\item \textbf{Ted Jacobson} and \textbf{Erik Verlinde}: Their revolutionary work on ``gravity as entropy force'' provided the thermodynamic pivot for this book's reconstruction of general relativity.

\item \textbf{E. H. Lieb} and \textbf{D. W. Robinson}: Their \textbf{Lieb-Robinson bound} provided solid evidence for the speed-of-light limit of this book's microscopic engine (QCA).
\end{itemize}

\section{Echoes of Philosophy}

Finally, I thank those thinkers who transcend time and space, who gave warmth to cold formulas:

\begin{itemize}
\item \textbf{Laozi}: Two thousand five hundred years ago, he used the language of ``Tao'' and ``circle'' to prophesy the ultimate truth we attempt to describe with quantum mechanics today.

\item \textbf{Douglas Hofstadter}: His \textit{Gödel, Escher, Bach} inspired this book's discussion of consciousness as a ``strange loop.''

\item \textbf{Lewis Carroll}: Thanks to his Red Queen, who provided the most vivid metaphor for describing the existential anxiety in an expanding universe.
\end{itemize}

\section{To the Universe}

Finally, I thank the universe itself.

Thank it for providing the generous budget of $c_{FS}$, allowing us to exist.

Thank it for designing the game between $v_{ext}$ and $v_{int}$, allowing us to experience time and mass.

Thank it for leaving the gap between $\pi$ and $\varphi$, allowing us to seek freedom within conservation, to seek ascension within cycles.

This book does not belong to me; it belongs to that unique, eternally rotating vector. I am merely a record it left at this time and place, through a brief self-reference of neurons.

May every reader, at the moment of closing this book, feel their deep connection with that great circle.

\textbf{Thank you.}

