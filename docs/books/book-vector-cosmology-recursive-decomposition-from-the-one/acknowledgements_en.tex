\chapter*{Acknowledgements: Standing on the Shoulders of Geometry}

The birth of this book \textbf{Vector Cosmology: The Conservation of the Circle} originated from a seed initially planted in an academic paper.

In December 2025, I wrote in Singapore that paper titled \textbf{``Time as Fubini-Study Arc-Length''}. At that time, I attempted to solve a long-standing puzzle in physics: how to find a unified geometric skeleton among three distinctly different ``times''---quantum mechanics, relativity, and thermodynamics. That paper is the rigorous mathematical core of this book, and this book is a free growth of that core in philosophy and logic.

As the author of this book, I (\textbf{Haobo Ma}) am deeply aware that the geometric edifice I have constructed does not belong solely to me. It is built upon foundations laid by countless giants of physics and mathematics over the past century. Here, I pay my deepest respects to them.

\section*{Mathematical Guides}

First, I thank \textbf{Guido Fubini} and \textbf{Eduard Study}. It was their definition of the \textbf{Fubini-Study metric} a century ago that provided us with a ``ruler'' for measuring changes in quantum states. Without this ruler, I could not define $c_{FS}$, nor derive that Pythagorean identity that governs this book.

I particularly thank \textbf{Y. Aharonov} and \textbf{J. Anandan}. Their groundbreaking work on geometric phases in 1987 convinced me that projective Hilbert space $P(\mathcal{H})$ is the true stage of physics. They pointed out the independence of geometric distance in quantum evolution, which directly inspired my redefinition of ``time'' as geometric arc length.

\section*{Founders of Physics}

In constructing physical mechanisms, I thank \textbf{Norman Levinson}. His \textbf{Levinson's theorem} proposed in 1949 is the mathematical soul of this book's core view that ``matter is counting.'' He showed us that particles are not solid entities, but topological knots tied by phases on the energy axis.

I thank \textbf{Elliott Lieb} and \textbf{Derek Robinson}. Their \textbf{Lieb-Robinson bound} provided solid causal law guarantees for this book's microscopic engine---Quantum Cellular Automata (QCA). It is precisely this speed limit that allows us to anchor the abstract FS capacity $c_{FS}$ with the physical world's speed of light $c$.

I also thank \textbf{E. P. Wigner} and \textbf{F. T. Smith} for their pioneering work in scattering time delay, which enabled us to transform ``time delay'' into ``geometric distance.''

\section*{Real-World Support}

I thank my colleagues and environment at \textbf{AELF PTE LTD, Singapore}. In this place full of computational power and innovative thinking, I was fortunate to fuse thoughts on distributed systems and AI architecture with explorations in fundamental physics. This interdisciplinary collision allowed me to re-examine ancient physical laws from the perspectives of ``computation'' and ``budget.''

I also thank all interlocutors (whether human or artificial intelligence) who provided inspiration during the formation of this theory. It was these continuous questions and deductions that ultimately evolved a simple formula $v_{ext}^2 + v_{int}^2 = c_{FS}^2$ into a complete worldview.

\section*{To the Unique Circle}

Finally, I thank the universe itself for its astonishing symmetry.

During the writing of this book, I often felt a sense of awe: seemingly complex relativistic effects and quantum phenomena could be unified so simply within the geometry of a circle. This further convinced me that \textbf{``Tao''} is not empty words; it is the highest summary of the universe's underlying minimal logic.

Although this book has ended, exploration has not ceased. In the first book, we depicted a closed, conserved circle, but is this the complete truth? Does that ``gap'' we temporarily ignored hint at grander secrets?

This will be the theme explored in \textbf{Vector Cosmology II: The Ascension of the Spiral}.

\textbf{Haobo Ma}

December 2025, Singapore

