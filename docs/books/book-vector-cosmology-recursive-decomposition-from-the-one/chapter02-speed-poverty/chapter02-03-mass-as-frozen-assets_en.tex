\section{Mass as Frozen Assets}

\begin{quote}
``Inertia is not an object's property of resisting motion, but the resistance of budget to reallocation.''
\end{quote}

In the geometric derivation of the Dirac Circle, we witnessed a mathematical miracle: the famous mass-energy relation $E^2 = p^2c^2 + m^2c^4$ in physics is essentially just an algebraic transformation of the information-velocity circle $v_{ext}^2 + v_{int}^2 = c_{FS}^2$.

However, this is not merely a mathematical game. This equation forces us to redefine one of the most fundamental yet perplexing concepts in physics---\textbf{Mass}.

In Newtonian mechanics, mass is regarded as an intrinsic property of matter, representing the ``amount of matter.'' But from the perspective of \textbf{Vector Cosmology}, this static definition is completely shattered. Mass is no longer a noun but a consequence of a \textbf{verb}.

\subsubsection{Static Budget and the Origin of Inertia}

If we view the universe as an economy, then \textbf{$c_{FS}$ (FS capacity)} is the cash flow you must spend at each moment. You cannot save it; you must let it flow---either to external space ($v_{ext}$) or to internal structure ($v_{int}$).

\textbf{What is mass ($m$)?}

Mass is the \textbf{``asset sedimentation''} formed when a system chooses to invest most of its budget in internal evolution ($v_{int}$).

When a particle possesses mass, it is essentially saying: ``I lock my $c_{FS}$ budget in the rotation of the internal dimension.'' This locking creates a strong \textbf{path dependence}, which we call \textbf{inertia} in macroscopic physics.

Why is it hard to push a massive object?

\begin{itemize}
\item \textbf{Traditional explanation}: Because it is heavy.

\item \textbf{Vector Cosmology explanation}: Because you are trying to force a system already running at full capacity to change its budget allocation table.

    A massive object at rest has its $v_{int}$ occupying almost 100\% of the $c_{FS}$ budget. When you try to push it (increase its $v_{ext}$), the system must painfully ``divest'' from its massive internal expenditure, converting part of $v_{int}$ into $v_{ext}$.

    This \textbf{``resistance to divestment''} is inertia. The larger the mass, the more internal budget is locked, and the greater the ``transaction cost'' (force) required to change the direction of this massive investment.
\end{itemize}

\subsubsection{The Eternity of Light: The Freedom of the Proletariat}

To truly understand the essence of mass, we must look at its opposite---\textbf{Light}.

Photons (and other massless particles) are the \textbf{proletariat} of the universe. Their position on the Dirac Circle is extreme:

\[m = 0 \implies v_{int} = 0\]

Substituting into the Pythagorean identity, we obtain:

\[v_{ext} = c_{FS}\]

This reveals three profound truths about light:

\begin{enumerate}
\item \textbf{Light must move at light speed}:

    Light does not ``want'' to move fast. Light is \textbf{penniless} (no internal mass to consume budget), so it is forced to pour all its $c_{FS}$ budget onto the external spatial axis. The speed of light $c$ is not a speed limit; it is the \textbf{full payout} when budget has nowhere else to go.

\item \textbf{Light has no time}:

    Because $v_{int} = 0$, the photon's internal clock never ticks. For a photon, from the moment it is born in the stellar core to the moment it strikes your retina, these two events are \textbf{simultaneous}. In light's subjective perspective, the universe is an instantaneous slice. Light gains absolute freedom in space at the cost of losing all experience in time.

\item \textbf{Light has no inertia}:

    Because light does not lock any budget in $v_{int}$, it has no ``fixed assets'' to maintain. But this does not mean light has no energy; its energy is entirely manifested as kinetic energy (frequency).
\end{enumerate}

\subsubsection{Geometric Preview of the Higgs Mechanism}

This raises an obvious question: If the universe's default state should be free flight like light ($v_{ext} = c_{FS}$), why would matter stop and become massive?

In the standard model of particle physics, this is called the \textbf{Higgs Mechanism}. In our geometric language, this can be described as a \textbf{``forced budget freeze''}.

You can imagine that in the very early universe, all particles were massless, all racing at $c_{FS}$. Suddenly, a ``sticky'' field (the Higgs field) appeared in space. Some particles interacted with this field. This interaction forced them to turn part of their $c_{FS}$ inward, beginning to spin in place (acquiring $v_{int}$).

This frequency of ``spinning in place'' is what we define as \textbf{mass}.

Thus, light-speed flight stopped, and the passage of time began. The original energy flow was curled into \textbf{matter}. Every atom is a segment of imprisoned light, spinning wildly in the internal dimension.

\subsubsection{Chapter Summary}

At this point, we have completed the first round of reconstruction of the macroscopic architecture.

\begin{itemize}
\item \textbf{Relativity} tells us the budget is limited.

\item \textbf{The Dirac Circle} tells us mass is the internal locking of budget.

\item \textbf{Light} tells us the eternal state after complete budget release.
\end{itemize}

The solid matter we see---tables, chairs, stars---are essentially \textbf{``frozen assets''}. The universe constructs a stable macroscopic world by locking $c_{FS}$ in microscopic internal cycles. But these assets have not truly disappeared; they are just burning intensely in ways we cannot see ($v_{int}$). Once these assets are unfrozen (nuclear reactions), the released $c_{FS}$ will once again shake the world.

Now that we understand how ``single particles'' form mass through budget allocation, the next question is: When these particles with massive $v_{int}$ assets gather together, how will they distort the rules of the entire market through ``monopoly''? In the next chapter, we will enter the realm of gravity.

