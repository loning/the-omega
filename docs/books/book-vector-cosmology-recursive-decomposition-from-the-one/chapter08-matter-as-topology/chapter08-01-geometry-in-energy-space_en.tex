\chapter{Matter as Topology}

We have seen that matter, at its deepest level, is counting pi $\pi$. But how is this counting physically accomplished?

In the macroscopic world, we are accustomed to viewing matter as static ``existence.'' But in the quantum world, the essence of matter is dynamic ``process.'' A stable atom is actually a wave that never stops, self-entangling. To understand this, we need to introduce a completely new geometric perspective to examine the most fundamental phenomenon in physics---\textbf{Scattering}.

In this chapter, we will leave the spatial axis we are familiar with and step into an abstract dimension---\textbf{Energy Space}. We will discover that so-called particle collisions and matter formation are not bounces on a billiard table but an elegant geometric dance performed by the universe's vector on the energy axis.

\section{Geometry in Energy Space}

\begin{quote}
``If you observe two particles colliding with a microscope, you won't see impact. You will see the wave function elegantly changing its phase pitch on the energy staff.''
\end{quote}

In classical mechanics, scattering is intuitive: two rigid balls collide and bounce apart. This is an event occurring in three-dimensional space ($x, y, z$). But in quantum mechanics, especially in the framework of \textbf{Vector Cosmology}, the essence of scattering is completely different.

We must abandon the ``billiard ball'' image and establish the ``phase space trajectory'' image.

\subsubsection{Trajectories on the Energy Axis}

Let us consider a single-particle scattering process (for example, an electron passing through the potential field near an atomic nucleus). In this process, the most crucial parameter is not time $t$ but \textbf{energy $\omega$}.

For each definite energy value $\omega$, the system has a corresponding scattering state $|\psi(\omega)\rangle$. This is a pure state vector living in projective Hilbert space $P(\mathcal{H})$.

When we scan from $\omega = 0$ to $\omega = \infty$ on the energy axis, this vector does not remain stationary. Due to the scattering phase shift $\delta(\omega)$ changing with energy, the vector $|\psi(\omega)\rangle$ traces a continuous curve in Hilbert space.

This is \textbf{geometry in energy space}.

\begin{itemize}
\item The universe no longer evolves in time but evolves in \textbf{energy}.

\item The object we study is the geometric shape of this trajectory $\omega \mapsto [\psi(\omega)]$ driven by the energy parameter.
\end{itemize}

\subsubsection{The Mirror of Schrödinger's Equation}

In the time domain, vector evolution is driven by the Hamiltonian $H$ ($|\dot{\psi}\rangle = -iH|\psi\rangle$).

In the energy domain, we discover a stunning mirror symmetry.

If we calculate the ``velocity'' of the vector changing with energy $\omega$---that is, the tangent vector $|\partial_\omega \psi\rangle$---we find its evolution equation has exactly the same form:

\[|\partial_\omega \psi(\omega)\rangle = i \tilde{Q}(\omega) |\psi(\omega)\rangle\]

Here, the ``generator'' driving evolution is no longer energy $H$ but a $\tilde{Q}$ called the \textbf{Wigner-Smith Time-Delay Operator}.

This is a profound duality:

\begin{itemize}
\item On the \textbf{time axis}, \textbf{energy ($H$)} drives phase rotation.

\item On the \textbf{energy axis}, \textbf{time ($Q$)} drives phase rotation.
\end{itemize}

Scattering processes are essentially the universe using ``time delay'' as a generator to draw geometric figures in energy space.

\subsubsection{FS Velocity as Distinguishability}

How fast does the vector move on this energy trajectory?

This requires our old friend---\textbf{Fubini-Study (FS) velocity}. But this time, it is defined on the energy parameter:

\[v_{FS}^{(\omega)} = ||\partial_\omega \psi(\omega)||_{FS}\]

According to derivations in the paper, this ``velocity in energy space'' has a clear physical meaning: \textbf{it strictly equals the standard deviation of the Wigner-Smith time-delay operator}.

\[v_{FS}(\omega) = \Delta Q(\omega)\]

What does this mean?

\begin{itemize}
\item If $v_{FS}$ is large, it means that with tiny energy changes, the scattering state undergoes dramatic changes (orthogonalization). This corresponds to physical \textbf{Resonance}.

\item Near resonance points, the vector rotates wildly in projective space, tracing huge arc lengths. It is precisely this dramatic geometric knotting that transforms a fleeting scattering state into a long-lived \textbf{``quasiparticle''}.
\end{itemize}

\subsubsection{Conclusion: No Collision, Only Winding}

Through the geometric perspective of energy space, we see through the illusion of ``collision.''

When two particles meet, they do not really ``collide.'' What actually happens is: the system's total vector, driven by the energy axis, undergoes a geometric evolution with extremely high curvature.

\begin{itemize}
\item \textbf{Free flight} is a smooth straight line.

\item \textbf{Scattering/collision} is a bend in the trajectory.

\item \textbf{Matter formation (bound states)} is the trajectory curling into a closed loop or dead knot.
\end{itemize}

So-called ``matter'' is those regions on the energy manifold where the universe's vector has \textbf{highest curvature and tightest winding}. The reason we feel atoms are ``hard'' is because at that point, the phase rotation speed $v_{FS}^{(\omega)}$ reaches its extreme, forming a geometrically difficult-to-untie topological structure.

