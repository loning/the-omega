\section{Micro-Parallelism Axiom: Time Is Not Merely a Linear Flow---It Has ``Thickness'' and ``Branching'' at the Microscopic Scale}

In standard physics, spacetime is modeled as a smooth four-dimensional manifold, where time $t$ is a one-dimensional real parameter. In this picture, the past has gone, the future has not arrived, and physical entities exist only on the infinitely thin slice of ``now,'' like walking a tightrope.

However, if we take seriously the proposition that ``the universe is computation,'' the above linear-time picture appears overly simplistic. In computer science, any non-trivial computation (logic gate operation) is not instantaneous. It requires \textbf{reading (Read)} input, \textbf{processing (Process)} state, and \textbf{writing (Write)} output. This means that at the microscopic logical scale, time has \textbf{structure}, even \textbf{thickness}.

This section will introduce a new axiomatic assumption---the \textbf{Micro-Parallelism Axiom}. We will prove that as long as we acknowledge that microscopic spacetime has a \textbf{layered} and \textbf{parallel} logical structure, the Standard Model's gauge group $SU(3) \times SU(2) \times U(1)$ will naturally emerge like spectral analysis.

\subsection{Axiom $\Sigma$: Computational Texture of Spacetime}

We refine the microscopic structure of QCA as follows:

\textbf{Axiom 2.2 (Micro-Parallelism Axiom)}:

Each macroscopic moment $t$ and position $x$ in the QCA network is not a single point in underlying logic, but a \textbf{multi-threaded computational complex}.

In the process of evolving one step, a fundamental physical entity (fermion) simultaneously occupies the following three orthogonal logical dimensions:

\begin{enumerate}
\item \textbf{Phase Dimension}: The complex argument $\theta$ of the wave function.

\item \textbf{Temporal Layers}: Computational \textbf{input buffer (Past/Input)} and \textbf{output buffer (Future/Output)}.

\item \textbf{Spatial Rails}: Three independent parallel computational paths corresponding to macroscopic spatial dimensions $x, y, z$.
\end{enumerate}

Based on this axiom, we can redefine fundamental interaction forces (gauge fields) as \textbf{local reference frame transformations on these three logical dimensions}.

\subsection{$U(1)_Y$: Phase Synchronization of Global Time Flow}

This is the most fundamental symmetry, corresponding to the complex nature of quantum mechanical wave functions.

In QCA networks, although space is discrete, to support unitary evolution and interference, the range of wave functions must be the complex domain $\mathbb{C}$. This means each node has an internal ``clock pointer'' (phase).

\begin{itemize}
\item \textbf{Symmetry Origin}: Phase dimension.

\item \textbf{Physical Mechanism}: To ensure all nodes' logical clocks maintain phase consistency across the entire network (i.e., the definition of moment $t$ is synchronized across the universe), the network must allow adjustment of local phases.

\item \textbf{Group Structure}: The rotation group of complex phases is $U(1)$.

\item \textbf{Corresponding Force}: \textbf{Electromagnetic force (photons)}.

\begin{itemize}
\item More precisely, weak hypercharge (Hypercharge $Y$).

\item Charge conservation is essentially conservation of \textbf{continuity of global time flow}. Photons are synchronization signals used to calibrate ``time phases'' at different locations.
\end{itemize}
\end{itemize}

\subsection{$SU(2)_L$: Time Branching and Input/Output Entanglement}

Next, we delve into the microscopic structure of time. Computation is not instantaneous; a logic gate $\hat{G}$ transforms input state $|\psi_{in}\rangle$ into output state $|\psi_{out}\rangle$. At the microscopic scale, particles must ``temporarily'' exist simultaneously in both states, forming a superposition.

We can model microscopic time as a two-layer structure:

\begin{itemize}
\item \textbf{Layer A (Input/Read/Past)}: Stores data at moment $t$.

\item \textbf{Layer B (Output/Write/Future)}: Stores precomputed data at moment $t+1$.

\item \textbf{Symmetry Origin}: Temporal layers.

\item \textbf{Physical Mechanism}: The unitary rotation group on this two-layer system is $SU(2)$. It allows particles to mix between the two microscopic slots of ``past'' and ``future.''

\item \textbf{Corresponding Force}: \textbf{Weak Interaction}.

\item \textbf{Solution to the Chirality Mystery}:

\begin{itemize}
\item Why does the weak force only act on left-handed particles?

\item In QCA dynamics, \textbf{chirality} encodes the direction of information flow. Typically, ``left-handedness'' corresponds to \textbf{``read mode''} (flowing from past to present), while ``right-handedness'' corresponds to \textbf{``write mode''} (flowing from present to future).

\item Non-trivial operations of logic gates (such as changing particle flavor) mainly occur during the \textbf{read phase} (when processing input data). Once data processing is complete and written to output (becoming right-handed), it becomes established fact and no longer participates in such deep flavor mixing.

\item Therefore, $SU(2)$ symmetry naturally breaks parity, attaching only to left-handed input flow.
\end{itemize}
\end{itemize}

\subsection{$SU(3)_C$: Spatial Parallelism and Three-Dimensional Interlocking}

Finally, we address the most complex strong interaction. Why is it $SU(3)$? Why do quarks have three colors?

This directly stems from the dimensionality $D=3$ of our macroscopic space. In microscopic computation, to achieve isotropic 3D movement, particles cannot simply ``jump'' once. They must run a \textbf{parallel algorithm}.

\begin{itemize}
\item \textbf{Symmetry Origin}: Spatial rails ($x, y, z$).

\item \textbf{Physical Mechanism}: A macroscopic particle (such as a quark) consists microscopically of three \textbf{``copies''} (Partons/Threads):

\begin{itemize}
\item Copy R (Red): Responsible for computing displacement in the $x$-axis direction.

\item Copy G (Green): Responsible for computing displacement in the $y$-axis direction.

\item Copy B (Blue): Responsible for computing displacement in the $z$-axis direction.
\end{itemize}

\item For these three copies to appear as a unified point-like particle macroscopically, they must maintain extremely strong \textbf{quantum coherence} among themselves.

\item \textbf{Group Structure}: The unitary mixing symmetry among these three independent channels ($R, G, B$) is precisely $SU(3)$.

\item \textbf{Corresponding Force}: \textbf{Strong Interaction (gluons)}.

\item \textbf{Geometric Explanation of Confinement}:

\begin{itemize}
\item What is ``color confinement''? It means nature only allows ``white'' states to exist.

\item ``White'' corresponds to balanced superposition of $R+G+B$. This means the particle's evolution in the $x, y, z$ directions is synchronized and complete.

\item If you try to pull out a single red quark (R), you are actually trying to strip out the \textbf{$x$-axis component} of a 3D object while discarding $y$ and $z$. Geometrically, this is equivalent to trying to create a \textbf{one-dimensional topological defect} (String) with only length but no width or height. This requires enormous energy, manifesting as tension in the gluon tube (Flux Tube).
\end{itemize}
\end{itemize}

\subsection{Summary: Fingerprints of Code}

Through the Micro-Parallelism Axiom, we find that the seemingly chaotic group structure $SU(3) \times SU(2) \times U(1)$ of the Standard Model is actually a \textbf{bottom-level hardware architecture diagram of the cosmic computer}:

\begin{itemize}
\item \textbf{$SU(3)$} tells us: The processor is \textbf{3-core parallel} (corresponding to 3-dimensional space).

\item \textbf{$SU(2)$} tells us: The pipeline is \textbf{2-level buffered} (corresponding to input/output logic).

\item \textbf{$U(1)$} tells us: The system is driven by a unified \textbf{complex clock} (corresponding to quantum phase).
\end{itemize}

The ``Grand Unified Theory'' (GUT) that physicists have been searching for half a century may not need to assume higher-dimensional mathematical symmetries (such as $SU(5)$), but only need to look back and examine the computational logic of spacetime itself. \textbf{Forces are projections of spacetime's logical dimensions.}

