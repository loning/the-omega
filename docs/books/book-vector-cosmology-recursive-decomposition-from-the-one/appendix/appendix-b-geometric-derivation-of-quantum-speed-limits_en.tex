\chapter{Geometric Derivation of Quantum Speed Limits}

In Chapter 2 ``The Poverty of Speed'' and Chapter 10 ``The Entropic Speed Limit Axiom'' of the main text, we repeatedly used a core conclusion: a system's evolution speed is limited by the variance of its energy (or generator). Physical changes cannot occur infinitely fast; they are constrained by strict \textbf{Quantum Speed Limits (QSL)}.

This appendix will provide rigorous mathematical derivations of these speed limits based on Fubini-Study geometry. We will prove that Mandelstam-Tamm type bounds are not merely manifestations of the energy-time uncertainty principle, but direct corollaries of the axiom in Riemannian geometry that ``the straight line is the shortest path between two points.''

\section{From Variance to Distance}

Consider a quantum evolution process described by parameter $\lambda$ (which can be any physical time parameter). Its state vector $|\psi(\lambda)\rangle$ follows the generalized Schrödinger equation:

\[\frac{d}{d\lambda}|\psi(\lambda)\rangle = -iK(\lambda)|\psi(\lambda)\rangle\]

where $K(\lambda)$ is the self-adjoint operator (generator) driving the evolution.

According to the conclusion of Appendix A, the instantaneous FS velocity along this trajectory strictly equals the standard deviation of the generator:

\[v_{FS}^{(\lambda)} = \Delta K(\lambda) = \sqrt{\langle K^2 \rangle - \langle K \rangle^2}\]

We want to calculate the \textbf{FS Path Length} that the system travels in projective Hilbert space from parameter $\lambda_0$ to $\lambda_1$. This can be obtained by integrating the velocity:

\[L_{FS}(\lambda_0, \lambda_1) = \int_{\lambda_0}^{\lambda_1} v_{FS}^{(\lambda)} d\lambda = \int_{\lambda_0}^{\lambda_1} \Delta K(\lambda) d\lambda\]

\section{Derivation of Geometric Bounds}

In Riemannian geometry, the shortest path connecting two points $[\psi(\lambda_0)]$ and $[\psi(\lambda_1)]$ is a geodesic. Therefore, the actual path length $L_{FS}$ traveled by the system must be greater than or equal to the \textbf{FS Distance} $d_{FS}$ between these two points:

\[d_{FS}([\psi(\lambda_0)], [\psi(\lambda_1)]) \le \int_{\lambda_0}^{\lambda_1} \Delta K(\lambda) d\lambda\]

If we assume that throughout the evolution process, the generator's variance has a maximum value $\Delta K_{max}$, we can bound the integral to obtain a simple inequality:

\[d_{FS} \le \Delta K_{max} \cdot |\lambda_1 - \lambda_0|\]

Rearranging this formula, we obtain a lower bound on the time interval:

\[|\lambda_1 - \lambda_0| \ge \frac{d_{FS}([\psi(\lambda_0)], [\psi(\lambda_1)])}{\Delta K_{max}}\]

This is the parameter-free geometric form of the famous \textbf{Quantum Speed Limit (QSL)}.

It tells us: \textbf{For a quantum system to change its state (i.e., travel distance $d_{FS}$), it must consume a product of ``variance resources'' ($\Delta K$) and ``time resources'' ($|\lambda_1 - \lambda_0|$).} If the system's energy variance is small (``poor''), it must spend a long time to complete the evolution.

\section{The Relationship Between Intrinsic Time and Laboratory Time}

Now, we apply this inequality to the core architecture of \textbf{Vector Cosmology}.

Introducing the definition of \textbf{Intrinsic Time $\tau$}, i.e., choosing parameters such that FS velocity is constant at the universe's total budget $c_{FS}$:

\[||\partial_{\tau}\psi(\tau)||_{FS} = c_{FS}\]

Using the chain rule $d\tau/d\lambda = v_{FS}^{(\lambda)}/c_{FS}$, we can establish a strict conversion relationship between any physical time parameter $\lambda$ (such as laboratory time $t$) and intrinsic time $\tau$:

\[d\lambda = \frac{c_{FS}}{\Delta K(\lambda)} d\tau\]

Integrating this relationship, we obtain the connection between the two time increments:

\[\lambda_1 - \lambda_0 = \int_{\tau_0}^{\tau_1} \frac{c_{FS}}{\Delta K(\lambda(\tau))} d\tau\]

This integral formula is the mathematical root of all ``time relativity'' phenomena in the book.

\begin{itemize}
\item \textbf{Time Dilation}: If $\Delta K$ (e.g., internal mass energy gap) decreases, the denominator decreases. To cover the same intrinsic distance $d\tau$, the required external time $d\lambda$ increases (time dilation).

\item \textbf{Photon's Eternity}: For photons, $\Delta K \to 0$ (in the mass sector), which means $d\lambda \to \infty$. That is, a finite instant in their own coordinate system corresponds to infinite time in the external world.
\end{itemize}

Through this derivation, we prove that relativistic effects are not the curvature of spacetime background, but inevitable consequences of systems following geometric conservation laws in the \textbf{``variance-time'' trading market}.

