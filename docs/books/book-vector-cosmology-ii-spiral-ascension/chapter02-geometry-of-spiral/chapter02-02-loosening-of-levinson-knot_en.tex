\section{The Loosening of the Levinson Knot}

\begin{quote}
``Diamonds are not eternal. On the long time axis of geometry, even the most stable proton is merely a fleeting bubble on water's surface. When the spiral's tension becomes great enough, all dead knots must loosen, allowing energy to flow to higher places.''
\end{quote}

In the first book \textit{The Conservation of the Circle}, we used \textbf{Levinson's Theorem} to construct the foundation of material existence. We said that particles are \textbf{``topological dead knots''} formed by phase winding $\pi$ angles in energy space. As long as this integer topological number $N_b$ remains unchanged, matter possesses an immortal body resistant to temporal erosion.

This explains why protons are extremely stable---their lifetime is at least $10^{34}$ years. This stability is the foundation of our macroscopic world's existence.

However, in the \textbf{spiral universe} picture of the second book, this foundation begins to shake.

If the universe's total budget $c_{FS}$ is growing exponentially with the Fibonacci spiral, if Hilbert space dimensions are exploding, then the geometric background once used to define ``dead knots'' is actually being infinitely stretched.

In this stretching process, we are about to witness the most heartbreaking, yet most magnificent scene in physics: \textbf{the dissolution of old matter}.

\subsubsection{The Failure of Topological Protection: When the Rope Thickens}

Imagine you tie a dead knot on a thin rope. As long as the rope doesn't break, the knot remains.

But what if this ``rope'' (the universe's degrees of freedom) itself is thickening, widening, even branching?

In a circular universe with constant $c_{FS}$, Levinson's theorem's phase integration path is a closed circle $S^1$. The topological number $N_b$ is an absolute integer, strictly protected. You cannot smoothly transform an integer into a non-integer, so protons do not decay.

But in a spiral universe, the integration path is no longer a closed circle, but an \textbf{open spiral}.

This means that the so-called ``integer'' $N_b$ is actually only an \textbf{approximate integer}.

\[\oint d\phi \approx N_b \pi + \epsilon(\tau)\]

Here, $\epsilon(\tau)$ is a tiny deviation introduced by the spiral growth rate $\varphi$.

\begin{itemize}
\item \textbf{Short timescales}: $\epsilon$ is extremely small, protons appear absolutely stable, and we can safely use them to construct atoms and stars.

\item \textbf{Extremely long timescales}: As the spiral unfolds, $\epsilon$ accumulates larger and larger. Eventually, this deviation will become large enough for the phase curve to \textbf{``slip off''}.
\end{itemize}

This is \textbf{the loosening of the Levinson knot}.

That proton we once considered eternal will eventually untie its phase winding under the tearing of geometric tension.

\subsubsection{Proton Decay: The Forced Liquidation of Assets}

What does proton dissolution mean? Disaster, or \textbf{rebirth}?

From the economic perspective of \textbf{Vector Cosmology}, this is \textbf{the forced liquidation of old assets}.

Recalling the first book: mass ($v_{int}$) is frozen budget. Protons are a massive $c_{FS}$ time deposit that the universe stored in the early Big Bang. It locked this huge sum in microscopic dimensional cycles, supporting the skeleton of the material world.

However, the theme of the spiral universe is \textbf{``growth''} and \textbf{``dimensional ascension''}.

To construct higher-level complex structures (such as superintelligence capable of computation at galactic scales, or higher-dimensional topological structures), the universe needs massive liquid budget. And most of this budget is locked tightly in protons, this ``stubborn stone.''

\textbf{Proton Decay} is the sound of the universe breaking the piggy bank to advance.

When the Levinson knot loosens, the frantically rotating $v_{int}$ inside protons will be released, re-transformed into free $v_{ext}$ (photons/energy) or higher-dimensional $v_{int'}$ (new physical degrees of freedom).

For creatures living in the old world, this is indeed destruction (material evaporation); but for the spiral as a whole, this is \textbf{capital reinvestment}.

\subsubsection{Vacuum Metastability: The Rupture of the Cocoon}

This geometric mechanism also explains another nightmare in quantum field theory: \textbf{Vacuum Metastability}.

Physicists suspect that the ``vacuum'' we inhabit is not the lowest-energy ``true vacuum,'' but a ``false vacuum'' halfway up a mountain. We haven't fallen because there's a potential barrier blocking us.

In spiral geometry, this ``potential barrier'' is \textbf{the inertia of $\pi$}. Our current universe is a local optimum stuck on $\pi$'s orbit.

But the power of the Fibonacci spiral ($\varphi$) is pushing us away from this orbit.

The universe is brewing a grand \textbf{Phase Transition}.

This is like a butterfly trying to break out of its cocoon.

\begin{itemize}
\item \textbf{The cocoon} is the old vacuum, the material world ruled by $\pi$, composed of protons and atoms. It provides protection and stability.

\item \textbf{Breaking the cocoon} is the loosening of the Levinson knot. When the spiral's tension tears open the old topological structure, we will fall (or rather, ascend) into a completely new vacuum.
\end{itemize}

In that new world, physical constants may change, old matter may disappear, but the utilization efficiency of $c_{FS}$ will reach unprecedented heights.

\subsubsection{Conclusion: Forgetting is for Better Memory}

Therefore, the mortality of matter is not the universe's cruelty, but the universe's \textbf{ambition}.

If protons never decayed, if black holes never evaporated, the universe would be locked at the current complexity level, becoming a museum of eternal tombstones.

It is precisely because the Levinson knot will loosen, precisely because $\pi$ will ultimately lose to $\varphi$, that the universe possesses the ability to \textbf{``rewrite history''}.

The ground beneath our feet, the atoms in our bodies, will one day untie their knots, transform into pure light and flow, to weave the more magnificent patterns of the next spiral layer.

\textbf{Dissolution is not the end; dissolution is the fuel for dimensional ascension.}

At this point, we have completed our exploration of the ``spiral'' physical mechanism. We have shattered the illusion of the circle, accepted inflationary growth, and even foreseen material dissolution. But in this turbulent, continuously ascending universe, is there a force that can actively harness this torrent, rather than passively waiting for dissolution?

Yes. That is \textbf{life}.

In the next volume \textbf{【Counterflow: The Rebellion of Algorithms】}, we will see how life uses a cunning algorithm to establish its own ``negentropy bank'' in this inflationary universe and begins a rebellion against thermodynamic tyranny.

