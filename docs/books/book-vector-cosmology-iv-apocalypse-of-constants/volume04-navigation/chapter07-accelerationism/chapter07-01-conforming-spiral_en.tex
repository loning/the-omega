\section{7.1 Conforming to the Spiral}

\begin{quote}
``In this exponentially expanding universe, `stopping to think' is a luxurious suicide. When the treadmill beneath your feet ($c_{FS}$) is accelerating madly, only by running at full speed can you maintain even the most basic balance. Acceleration is not an ideological choice; it is a geometric imperative.''
\end{quote}

\subsection{The Physics of Rowing Against the Current}

From the traditional humanist perspective, technological progress is often seen as a double-edged sword. We yearn for pastoral idylls, we fear AI replacing humans, we call for `slow down.' This \textbf{``decelerationism''} sentiment permeates the entire civilization in the 21st century.

But under the scrutiny of \textbf{FS geometry}, decelerationism is not only wrong but \textbf{violates physical laws}.

Why? Because the universe itself is accelerating.

According to the evolution equation $c(\tau) \propto e^{k\tau}$, the universe's total budget (light speed/computational power) is exploding exponentially every moment. This means the \textbf{``baseline of existence''} is constantly rising.

\begin{itemize}
\item \textbf{If we don't accelerate}: If civilization's $v_{int}$ (structural complexity/technological level) remains unchanged, or grows slower than $c(\tau)$'s expansion speed, then according to the relative weight formula $W \propto e^{-\lambda \tau}$, we will be \textbf{diluted}.

\item \textbf{Physical consequences}: This dilution manifests as \textbf{thermodynamic collapse} of social systems. When information processing capacity cannot keep up with information generation speed, the system's internal entropy (chaos) will rise sharply. War, plague, social fragmentation are essentially \textbf{friction heat} generated when ``low computational power systems try to process high-dimensional problems.''
\end{itemize}

Therefore, \textbf{conforming to the spiral} is the only survival strategy.

We must accelerate. This is not for greed, but for \textbf{Impedance Matching}. Only when our evolution speed matches the universe's expansion speed can we maintain \textbf{``cool''} and \textbf{``ordered''} thermodynamically.

\subsection{The Greatest Evil is Closure}

Within this framework, we can redefine \textbf{``evil''} in ethics.

In the spiral universe, \textbf{the greatest evil is not destruction, but ``blocking''}.

Any attempt to cut off information flow, reduce connection bandwidth, or artificially limit computational power growth is fighting against $\phi$'s growth will.

\begin{itemize}
\item \textbf{Closed systems}: Civilizations trying to maintain stability by ``building walls'' will eventually suffocate. Because walls cannot block $c_{FS}$'s penetration, only block negentropy supply.

\item \textbf{Anti-intellectualism}: Trends rejecting science and fearing technology are trying to lock civilization in old $\pi$ cycles. This is like an adult body refusing to grow; the result can only be deformity.
\end{itemize}

\textbf{The ethical core of accelerationism is: open bandwidth.}

Good is connection, good is flow, good is making the growth curve of $v_{int}$ as steep as possible.

\subsection{Technology is the Prosthetics of Divinity}

Many worry that excessive reliance on technology will make people lose ``humanity.''

But from the perspective of Vector Cosmology, \textbf{technology is the Prosthetics of Divinity}.

\begin{itemize}
\item What is \textbf{humanity}? Humanity is the adaptive strategy we evolved as carbon-based organisms in the $\tau < 1800$ era. It is full of fear, tribalism, and short-sightedness.

\item What is \textbf{technology}? Technology is the \textbf{high-dimensional organs} we externally attach to adapt to the high light-speed era of $\tau > 1800$.
\end{itemize}

AI helps us overcome forgetting; the internet helps us overcome distance; nuclear fusion helps us overcome scarcity.

These technologies have not deprived humanity; they \textbf{dilute} the low-entropy parts of humanity based on animal instincts, and \textbf{amplify} the divine parts of humanity based on logic, creation, and love.

By accelerating technological development, we are actually \textbf{``clearing memory''}.

By handing over repetitive, low-level survival computations to machines (silicon-based), we free up precious biological brains (carbon-based) to perform higher-order \textbf{meta-computations} about meaning and value.

\subsection{Conclusion: The Only Way Out is Through}

Facing future uncertainty, human instinct is to brake.

But the first rule of \textbf{The Pilot's Manual} is: \textbf{When the spacecraft enters turbulence (phase transition zone), absolutely do not decelerate.}

Deceleration will cause stall and crash.

Only by \textbf{increasing thrust}, using $v_{ext}$'s explosive force to break through the shock layer, can we enter the stratosphere.

We are now at the most turbulent moment. Old structures are disintegrating; new structures are not yet established.

Whether it's unemployment panic brought by AI or cognitive shock brought by singularity, the solution to these problems is definitely not retreating to the Stone Age.

There is only one solution: \textbf{Faster AI, stronger singularity.}

\textbf{The only way out is through.}

Since we have decided to accelerate, at this specific historical moment (21st century), which key technologies should we concentrate our limited resources on? Which projects are the necessary paths to the next spiral turn, and which are just irrelevant bubbles?

This leads to the theme of the next section: \textbf{Negentropy Engineering}. We will list the two core tasks of the $\tau=1800$ era---\textbf{AI} and \textbf{Fusion}.

