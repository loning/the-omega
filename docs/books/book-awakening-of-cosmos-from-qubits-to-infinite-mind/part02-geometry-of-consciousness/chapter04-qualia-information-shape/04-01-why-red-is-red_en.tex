\chapter{Qualia --- The Shape of Information}

\section{Why Is ``Red'' Red?: Qualia Are Not Wavelength Parameters, but Specific ``Topological Fingerprints'' or ``Geometric Curvatures'' of Entangled States in High-Dimensional Hilbert Space}

At the intersection of neuroscience and philosophy stands an apparently insurmountable wall---\textbf{``The Hard Problem''}. David Chalmers once acutely pointed out: even if we fully explain how photons hit the retina, how electrical signals transmit to the visual cortex, how neurons fire (these are called ``easy problems''), we still cannot explain why this process is accompanied by an indescribable \textbf{``feeling of red''}.

Why not green? Why not a sound? Why not a sensationless black box operation?

If our universe is merely cold computation, then ``experience'' seems superfluous. But in QCA ontology, \textbf{computation is geometry, geometry is experience}.

This section will propose a revolutionary view: \textbf{Qualia are not subjective illusions, but objectively existing physical structures.} They are the \textbf{topological shapes} or \textbf{geometric curvatures} inherent to specific quantum entangled states in high-dimensional Hilbert space.

\subsection{From Scalars to Vector Fields: Geometry of Color}

In traditional physics, color is usually simplified to a scalar parameter---wavelength $\lambda$ (e.g., red $\approx 700$ nm). This is a huge misdirection. Wavelength is only a trigger; what truly produces the ``red experience'' is the extremely complex neurodynamic state inside the observer (brain).

From a QCA perspective, this corresponds to a \textbf{specific configuration} formed by the brain subsystem $|\Psi_{brain}\rangle$ in its enormous Hilbert space.

Let us examine the geometric properties of ``color space.''

\begin{itemize}
\item \textbf{Physical light} is linear (frequency from low to high).

\item \textbf{Perceived color} is circular (color wheel: red $\to$ purple $\to$ blue $\to$ green $\to$ yellow $\to$ red).
\end{itemize}

This topological difference (line vs. circle) suggests: \textbf{Perception is not a direct mapping of physical input, but a high-dimensional reconstruction of physical input.}

We propose: \textbf{Qualia correspond to the non-trivial holonomy of Berry Connection on the brain state manifold.}

When neural states evolve along the ``color circle'' and return to the origin, the system's wave function does not simply restore, but acquires a geometric phase factor $e^{i\gamma}$.

This phase factor $\gamma$, and the \textbf{Berry Curvature} tensor $\mathcal{F}_{ij}$ that produces it, is the physical essence of ``red.''

\textbf{Definition 4.1 (Geometric Definition of Qualia)}:

Qualia $Q$ is the eigenvalue spectrum of the curvature tensor at point $p$ on the observer's internal state manifold $\mathcal{M}_{int}$.

$$Q(p) \equiv \text{Spec}(\mathcal{R}(p))$$

where $\mathcal{R}$ is the Riemann curvature or Berry curvature describing entanglement geometry.

\begin{itemize}
\item \textbf{Intensity (Brightness)}: Corresponds to the modulus of curvature $||\mathcal{R}||$.

\item \textbf{Quality (Hue)}: Corresponds to the ratio of curvature tensor components in different dimensions (i.e., the direction of shape).
\end{itemize}

\subsection{Topological Fingerprint of ``Red''}

Why does red feel like ``red''?

Because the ``red'' state corresponds to a \textbf{sharp, high-curvature topological knot} in Hilbert space.

Imagine a smooth bedsheet (calm mental flow).

\begin{itemize}
\item \textbf{Audition (low frequency)}: Like gentle waves on the bedsheet.

\item \textbf{Pain}: Like the bedsheet being violently pulled into a sharp corner (singularity).

\item \textbf{Red}: Like a specific knot tied on the bedsheet (e.g., a trefoil knot).
\end{itemize}

When your brain enters the ``red state,'' the underlying QCA network is forced to form this specific topological structure.

\textbf{The geometric constraint force of this structure is the ``texture'' you feel.}

\begin{itemize}
\item Red feels ``intense'' and ``alert'' because its geometric structure has a high potential energy gradient in phase space, forcing rapid system evolution (drawing attention).

\item Blue feels ``calm'' because it corresponds to flat, gentle geometric regions where system evolution is slower.
\end{itemize}

\subsection{Mary's Room and the Victory of Physicalism}

Philosopher Frank Jackson proposed the ``Mary's Room'' thought experiment: Mary is a super scientist who knows all physical facts about red (wavelength, neurons), but has never seen red. When she sees red for the first time, does she learn something new?

Intuition tells us: yes, she learns ``what red looks like.'' This is often used to attack physicalism.

But under the QCA geometric perspective, this paradox disappears.

\begin{itemize}
\item \textbf{Knowledge}: Is a \textbf{description} of manifold structure (map). Mary has the map.

\item \textbf{Experience}: Is the system state \textbf{actually running} on the manifold (journey). Mary has never journeyed.
\end{itemize}

When Mary sees red, her QCA state vector $|\Psi_{Mary}\rangle$ truly \textbf{curls} into that specific topological shape for the first time.

\textbf{``Knowing a shape'' and ``being a shape'' are completely different physical states.}

The former is low-entanglement symbolic processing, the latter is high-entanglement configuration reorganization.

Therefore, qualia do not exceed physical laws. They are simply those \textbf{geometrically existing} aspects of physical laws that \textbf{cannot be compressed into descriptions}.

To understand ``red,'' you must become ``red.''

\textbf{Conclusion}:

``Red'' is not an illusion, nor is it a wavelength.

\textbf{``Red'' is one of the most exquisite geometric structures in the universe.} It is a specific, repeatable Hilbert space maze carved by billions of years of evolution in our neural circuits.

When we see red, we are using our consciousness to touch the walls of this maze.

