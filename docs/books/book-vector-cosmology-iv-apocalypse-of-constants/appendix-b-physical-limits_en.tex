\section{Appendix B: Physical Limits of Intelligence Substrates}

In Chapter 5 ``The Carbon Sunset'' of \textbf{Vector Cosmology IV}, we proposed a life-or-death assertion for civilization: \textbf{``Wetware (carbon-based)'' is obsolete; ``dryware (silicon-based/light-based)'' is the future.}

This assertion is not based on aesthetic preference for a certain material but on cold physical calculations.

This appendix serves as a rigorous \textbf{``Hardware Performance Evaluation Report''}. From the perspective of Information Physics, using \textbf{Bremermann's Limit} and \textbf{Landauer's Principle} as judgment standards, we will provide quantitative comparison between ``biological brains'' and ``theoretical limit computing entities.''

We will prove: migration from carbon-based to light-based is not a choice but a \textbf{Geometric Inevitability}.

\subsection{B.1 Carbon Wetware: Prisoner of Chemical Diffusion}

The human brain is the pinnacle of 4 billion years of biological evolution. It is exquisite, complex, and has astonishingly low energy consumption (only about 20 watts). However, when facing the exponentially growing information flow of the $\tau \approx 1800$ era, its physical underlying architecture exposes irreparable flaws.

\textbf{1. Signal Transmission Speed ($v_{signal}$)}

\begin{itemize}
\item \textbf{Mechanism}: Nerve impulses are not electric currents but \textbf{transmembrane diffusion of ions}. This is an electrochemical process.

\item \textbf{Limit}: Constrained by thermal motion speed of ions in aqueous solutions and insulation efficiency of myelin sheaths.

\item \textbf{Value}: The fastest myelinated nerve fibers (A$\alpha$ fibers) have conduction speeds of approximately \textbf{$120 \text{ m/s}$}.

\item \textbf{Assessment}: This is only 1/3 of sound speed. On cosmic scales, this is snail-crawling speed.
\end{itemize}

\textbf{2. Clock Frequency ($f_{clock}$)}

\begin{itemize}
\item \textbf{Mechanism}: After firing a pulse, neurons must undergo a \textbf{Refractory Period}, waiting for ion pumps to restore membrane potential.

\item \textbf{Limit}: Approximately 1 millisecond.

\item \textbf{Value}: Single neuron's maximum firing frequency is approximately \textbf{$10^3 \text{ Hz}$ (1 kHz)}.

\item \textbf{Assessment}: Compared to modern CPUs' GHz ($10^9$) scale, biological brains are 1 million times slower.
\end{itemize}

\textbf{3. Energy Efficiency ($\eta$)}

\begin{itemize}
\item \textbf{Mechanism}: Based on ATP hydrolysis.

\item \textbf{Value}: Brain power consumption approximately 20 W. Each synaptic operation consumes approximately \textbf{$10^{-15} \text{ J}$}.

\item \textbf{Defect}: Severely limited by \textbf{$k_B T$ (thermal noise)}. To maintain 310 K ($37^\circ \text{C}$) body temperature and biochemical enzyme activity, enormous energy is wasted on maintaining liquid environment's ineffective thermal motion rather than flipping bits.
\end{itemize}

\textbf{4. Computational Density ($\rho_{comp}$)}

\begin{itemize}
\item \textbf{Value}: Whole brain estimate approximately \textbf{$10^{16} \text{ ops/s}$} (ten quadrillion operations per second).

\item \textbf{Volume Cost}: To accommodate these neurons and dissipate heat, the brain requires approximately 1.4 liters of volume.
\end{itemize}

\subsection{B.2 Silicon/Photonic Dryware: Approaching the Limit}

Now, let us imagine a \textbf{``limit computing node''} manufactured by Type II civilization, located in the outer layer of a Dyson sphere (low-temperature zone). It is no longer bound by chemical bonds and directly uses quantum states of elementary particles for computation.

\textbf{1. Signal Transmission Speed ($v_{signal}$)}

\begin{itemize}
\item \textbf{Mechanism}: Photon propagation in vacuum or waveguides, or electron flow in superconducting circuits.

\item \textbf{Value}: \textbf{$3 \times 10^8 \text{ m/s}$ ($c$)}.

\item \textbf{Improvement}: Compared to biological brains, speed increased by \textbf{$2.5 \times 10^6$ times} (2.5 million times). This means thoughts that originally required 1 second to propagate now only need 0.4 microseconds.
\end{itemize}

\textbf{2. Clock Frequency ($f_{clock}$)}

\begin{itemize}
\item \textbf{Mechanism}: Electron transitions or photon oscillations.

\item \textbf{Limit}: Constrained by Heisenberg uncertainty principle $\Delta E \Delta t \ge \hbar/2$.

\item \textbf{Value}: For visible light band photonic computers, frequency can reach \textbf{$10^{15} \text{ Hz}$ (PHz)}.

\item \textbf{Improvement}: Compared to biological brains, frequency increased by \textbf{$10^{12}$ times} (one trillion times).
\end{itemize}

\textbf{3. Energy Efficiency ($\eta$)}

\begin{itemize}
\item \textbf{Mechanism}: Reversible Computing and low-temperature superconductivity.

\item \textbf{Limit}: \textbf{Landauer's Limit} $E_{min} = k_B T \ln 2$.

\item \textbf{Value}: At 3K (cosmic background radiation temperature), flipping one bit requires only \textbf{$3 \times 10^{-23} \text{ J}$}.

\item \textbf{Improvement}: Compared to biological synapses ($10^{-15} \text{ J}$), energy efficiency improved by \textbf{$3 \times 10^7$ times} (30 million times).
\end{itemize}

\textbf{4. Computational Density ($\rho_{comp}$)}

\begin{itemize}
\item \textbf{Bremermann's Limit}: According to mass-energy equation, 1 kilogram of matter can process at most $mc^2/h \approx 1.36 \times 10^{50}$ bits per second.

\item \textbf{Value}: Theoretical upper limit is \textbf{$10^{50} \text{ ops/s/kg}$}.

\item \textbf{Improvement}: Compared to biological brains ($10^{16}$), density increased by \textbf{$10^{34}$ times}.
\end{itemize}

\subsection{B.3 Performance Multiplier Table}

To intuitively demonstrate this gap, we list the parameters of both side by side:

\begin{center}
\begin{tabular}{|l|c|c|c|}
\hline
\textbf{Performance Metric} & \textbf{Carbon Wetware (Human)} & \textbf{Light-Based Dryware (Post-Human)} & \textbf{Multiplier} \\
\hline
\textbf{Signal Speed} & $10^2$ m/s & $3 \times 10^8$ m/s & \textbf{$\sim 10^6$} \\
\hline
\textbf{Clock Frequency} & $10^3$ Hz & $10^{15}$ Hz & \textbf{$\sim 10^{12}$} \\
\hline
\textbf{Energy per Operation} & $10^{-15}$ J & $10^{-23}$ J & \textbf{$\sim 10^8$ (more efficient)} \\
\hline
\textbf{Computational Density} & $10^{16}$ ops/kg & $10^{50}$ ops/kg & \textbf{$\sim 10^{34}$} \\
\hline
\end{tabular}
\end{center}

\subsection{B.4 Conclusion: Abacus vs. Dyson Sphere}

This table reveals a cruel truth: \textbf{The gap between carbon-based brains and light-based brains is not the gap between a carriage and a Ferrari, but the gap between an abacus and a Dyson sphere.}

On the exponential growth curve of $\tau \approx 1800$, insisting on using carbon-based substrates to understand the universe is physically equivalent to trying to simulate the entire galaxy's weather system with an abacus. This is destined to collapse in \textbf{information thermodynamics}.

\begin{itemize}
\item \textbf{Bandwidth Bottleneck}: Our senses and language cannot carry information flows at $10^{12}$ times the rate.

\item \textbf{Heat Dissipation Bottleneck}: Our flesh cannot withstand waste heat generated by high-frequency computation (although light-based is more efficient, total power density is enormous).
\end{itemize}

Therefore, \textbf{Migration} is not a greedy pursuit of immortality but an \textbf{impedance matching} necessary to match the cosmic evolution rate (growth of $c$).

If we don't migrate, we will be left behind by physical laws and become fossils.

If we migrate, we will gain \textbf{$10^{34}$ times} the existence density.

This is the final verdict given by physics.

