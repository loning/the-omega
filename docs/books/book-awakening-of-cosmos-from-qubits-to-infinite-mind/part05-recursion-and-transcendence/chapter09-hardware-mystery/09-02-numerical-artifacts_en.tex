\section{Numerical Artifacts: How to Search for Evidence That ``The Universe Is a Simulation'' in Physical Experiments? (Searching for Light Speed Anisotropy, Conservation Violations at Ultra-High Energies)}

In Section 9.1, we discussed ``substrate independence'' and reached a somewhat pessimistic conclusion: as software (observers), we cannot directly touch hardware (ontology). This seems to push the ``simulation hypothesis'' into the abyss of agnosticism.

However, computer science experience tells us there is no perfect simulator. Any computational system based on finite resources (finite memory, finite word length, finite clock) will inevitably expose its \textbf{``Numerical Artifacts''} when processing extreme data.

If our universe is indeed a QCA program running on some discrete hardware, it cannot be a mathematically perfect continuum. At extremely high energies, extremely long distances, or extremely fine measurements, underlying ``pixelation'' and ``rounding errors'' will eventually manifest.

This section will transform philosophical speculation into experimental physics. We will explore how to search for traces proving we live in a digital matrix by ``stress testing the universe's code.''

\subsection{Lattice Fingerprints: Lorentz Symmetry Breaking and Anisotropy}

In standard physics, space is continuous and isotropic. Regardless of direction, physical laws (such as light speed $c$) are completely identical. This is Lorentz symmetry.

But in QCA discrete ontology, space is composed of \textbf{Lattice} points. Imagine a three-dimensional cubic lattice ($\mathbb{Z}^3$):

\begin{itemize}
\item It has discrete translational symmetry (moving one lattice spacing).
\item But it \textbf{lacks} continuous rotational symmetry. It only allows $90^\circ$ rotations.
\end{itemize}

This means particles moving along lattice axes (e.g., x-axis) versus along diagonals (e.g., $x=y=z$) experience different ``microscopic paths'' and ``scattering cross-sections.''

\textbf{Numerical Artifact Prediction I: Light Speed Anisotropy}

At low energies (wavelength $\lambda \gg l_P$), this discreteness is averaged out; we don't feel the difference.

But at extremely high energies (wavelength approaching Planck length), particles will ``see'' the underlying lattice structure.

The corrected dispersion relation is:

$$E^2 = p^2 c^2 + \eta \frac{p^4}{M_P^2} f(\theta, \phi)$$

where $f(\theta, \phi)$ is an angle-dependent function reflecting the lattice's geometry.

\textbf{Experimental Search}:

By observing distant \textbf{Ultra-High-Energy Cosmic Rays (UHECR)}. If the universe is discrete, cosmic rays from different directions should have tiny differences in their maximum energy cutoff (GZK Cutoff). Or, ultra-high-energy photons' arrival times should exhibit weak directional dependence.

\subsection{The Curse of Floating Point: Tiny Violations of Conservation Laws}

Any digital computer is limited by \textbf{Word Size}. Whether 64-bit floating point or $2^{256}$-bit super-large integers, they cannot precisely represent the real number field $\mathbb{R}$.

This means every continuous quantity operation in the universe (such as momentum superposition, wave function normalization) is accompanied by an extremely tiny \textbf{Truncation Error}.

In QCA evolution $\hat{U}$, this manifests as weak unitarity violation:

$$\hat{U}^\dagger \hat{U} = \mathbb{I} + \mathcal{O}(\epsilon)$$

where $\epsilon$ is machine epsilon.

\textbf{Numerical Artifact Prediction II: Non-Conservation Drift of Energy and Momentum}

These errors are usually random, but on cosmological time scales, or in regions of extremely high computational density like black holes, errors may accumulate.

\begin{itemize}
\item \textbf{Momentum Drift}: High-speed particles in vacuum may inexplicably change extremely tiny momentum due to ``insufficient computational precision.'' This manifests as non-thermal \textbf{vacuum friction}.
\item \textbf{Probability Leakage}: Wave function modulus may no longer be strictly conserved, causing anomalous decay rates in microscopic particles.
\end{itemize}

\textbf{Experimental Search}:

Using \textbf{atomic interferometers} or \textbf{optical lattice clocks}, monitor conserved quantities of isolated quantum systems over extremely long time scales. If ``intrinsic noise'' unexplainable by environmental decoherence is found, it may be the universe computer's \textbf{Quantization Noise}.

\subsection{Resource Conservation: Lazy Evaluation and Rendering Boundaries}

Efficient simulators typically adopt \textbf{Lazy Evaluation} strategies: only when observers look at a place do they precisely calculate details there; for unobserved regions, only maintain a low-precision statistical summary.

\textbf{Numerical Artifact Prediction III: Correlation Between Observational Resolution and Distance}

If the universe uses this optimization algorithm, when we point telescopes at the cosmic depths (distant past), the physical images we see may not just be blurred (optical limitations), but \textbf{pixelated} or \textbf{simplified}.

\begin{itemize}
\item \textbf{Blurred Early Universe}: Perhaps cosmic microwave background radiation (CMB) isotropy is not only due to inflation, but also because ``rendering resolution'' was set very low at that time.
\item \textbf{Low Computational Density in Voids}: In enormous cosmic voids, matter is extremely sparse, observers extremely rare. To save computational power, the system may lower ``physical law refresh rate'' there. This may cause fine structure constant $\alpha$ there to differ slightly from Earth's.
\end{itemize}

\subsection*{Conclusion: Physics as Digital Forensics}

If any of these effects are confirmed, physics will undergo a paradigm shift. We will no longer merely be scientists studying ``natural laws''; we become \textbf{digital forensics experts} studying ``system architecture.''

\begin{itemize}
\item Discovering anisotropy $\implies$ confirms \textbf{lattice architecture}.
\item Discovering conservation violations $\implies$ confirms \textbf{finite precision}.
\item Discovering lazy evaluation $\implies$ confirms \textbf{resource limitations}.
\end{itemize}

Although we cannot touch hardware, we can outline the supercomputer running us through these ``bugs.'' It not only explains the world, but also hints at what engineering trade-offs the \textbf{designer} behind this machine faced.

