\appendix
\chapter{Fubini-Study Metric and Velocity Formula}

In the main text, we constructed a grand physical picture: the universe is a vector evolving in projective Hilbert space, constrained by a constant information-velocity budget. To ensure this picture does not remain merely at the level of philosophical metaphor, we need to show its underlying mathematical skeleton.

This appendix will define the Fubini-Study (FS) metric in detail, derive the FS velocity formula, and clarify the strict quantitative relationship between it and the microscopic Lieb-Robinson velocity. This is the mathematical foundation supporting the ``Pythagorean identity'' throughout the book.

\section{Projective Space and Distance Definition}

The physical state space we discuss is not ordinary Hilbert space $\mathcal{H}$, but \textbf{Projective Hilbert Space ($P(\mathcal{H})$)}. This is because in quantum mechanics, two state vectors $|\psi\rangle$ and $e^{i\theta}|\psi\rangle$ differing only by a global phase factor represent the same physical state.

The FS metric is the unique natural, unitary-invariant Riemannian metric on $P(\mathcal{H})$. For two normalized pure state vectors $|\psi\rangle$ and $|\phi\rangle$ in $\mathcal{H}$ (i.e., $\langle\psi|\psi\rangle = \langle\phi|\phi\rangle = 1$), the \textbf{FS distance} between their corresponding points $[\psi]$ and $[\phi]$ in projective space is defined as:

\[d_{FS}([\psi], [\phi]) = \arccos\left( |\langle\psi|\phi\rangle| \right)\]

This distance has intuitive geometric meaning: it measures the ``distinguishability'' between two quantum states. If two states completely coincide ($|\langle\psi|\phi\rangle|=1$), the distance is 0; if two states are orthogonal ($|\langle\psi|\phi\rangle|=0$), the distance is $\pi/2$.

\section{The Relationship Between FS Velocity and Variance}

Consider a differentiable curve $\lambda \mapsto |\psi(\lambda)\rangle$ parameterized by $\lambda$. This represents the universe's evolution trajectory over time. On this trajectory, the \textbf{FS velocity} $v_{FS}^{(\lambda)}$ is defined as the FS norm of the tangent vector:

\[v_{FS}^{(\lambda)} = \left\| \frac{d}{d\lambda}\psi(\lambda) \right\|_{FS}\]

The specific calculation formula is:

\[||\partial_{\lambda}\psi||_{FS}^2 = \langle\partial_{\lambda}\psi|\partial_{\lambda}\psi\rangle - |\langle\psi|\partial_{\lambda}\psi\rangle|^2\]

In quantum mechanics, evolution is usually driven by a self-adjoint operator (generator) $K(\lambda)$ satisfying the Schrödinger equation form:

\[\frac{d}{d\lambda}|\psi(\lambda)\rangle = -iK(\lambda)|\psi(\lambda)\rangle\]

Substituting this into the velocity formula, we obtain a crucial physical conclusion: \textbf{FS velocity strictly equals the standard deviation (uncertainty) of the generator}.

\[v_{FS}^{(\lambda)} = \Delta K(\lambda) = \sqrt{\langle K^2 \rangle - \langle K \rangle^2}\]

This explains why we repeatedly emphasize ``velocity is variance'' in the main text. When the universe evolves, how fast it runs in geometric space depends entirely on the energy fluctuation $\Delta H$ of its driving Hamiltonian. For eigenstates ($\Delta H = 0$), FS velocity is zero, geometric time stops---this is the mathematical definition of ``static.''

\section{Physical Units and Calibration with Lieb-Robinson Velocity}

Throughout the book, we use an abstract constant $c_{FS}$ as the universe's total budget. In actual physical models, this constant is not arbitrarily chosen but determined by microscopic discrete structure.

In a \textbf{Quantum Cellular Automaton (QCA)} model with lattice spacing $a$ and update time step $\Delta t$, the maximum physical speed of information propagation in space (Lieb-Robinson velocity) is:

\[v_{LR}^{(phys)} = \frac{a}{\Delta t}\]

Correspondingly, the maximum FS velocity in projective Hilbert space (i.e., our $c_{FS}$) is calibrated as:

\[c_{FS}^{max} = \frac{2\pi}{\Delta t}\]

The relationship between them is:

\[c_{FS}^{max} = \frac{2\pi}{a} v_{LR}^{(phys)}\]

This relationship reveals dimensional conversion:

\begin{itemize}
\item $v_{LR}^{(phys)}$ is \textbf{spatial velocity} (meters/second).

\item $c_{FS}$ is \textbf{information capacity} or \textbf{frequency} (1/second).
\end{itemize}

When we speak of ``speed of light limit,'' geometrically we actually mean the universe's ``maximum information update frequency'' is finite. In the derivations of the main text, for simplicity, we usually choose natural units ($\hbar=1, a=1$), making $c_{FS}$ proportional to $v_{LR}$ in value, thus unifying macroscopic and microscopic descriptions.

\section{Intrinsic Time $\tau$}

Finally, we define the \textbf{Intrinsic Time $\tau$} used throughout the book. This is a special parameterization choice that makes the FS velocity along the trajectory constant at $c_{FS}$:

\[||\partial_{\tau}\psi(\tau)||_{FS} \equiv c_{FS}\]

Under this parameterization, the relationship between any other physical parameter $\lambda$ (such as laboratory time $t$) and intrinsic time $\tau$ is given by the chain rule:

\[\frac{d\tau}{d\lambda} = \frac{v_{FS}^{(\lambda)}}{c_{FS}} = \frac{\Delta K(\lambda)}{c_{FS}}\]

This formula is the mathematical parent of all ``time dilation'' and ``time delay'' effects in the book. It tells us: the rate at which any physical clock runs depends on that clock's ability to consume budget $\Delta K$ in FS geometry.

