\chapter*{Preface: The Natural Generation}

In the journey of the \textbf{Vector Cosmology} trilogy, we have traveled a long road.

In the first book \textbf{The Conservation of the Circle}, we stood in the sanctuary of \textbf{$\pi$} and saw a perfect geometric universe constructed from Fubini-Study metrics and Pythagorean identities. That was a story about the \textbf{``Body''}---the universe's skeleton is rigid, conservation laws are absolute, matter is dead knots of phase. We marveled at that static beauty of order.

In the second book \textbf{The Ascension of the Spiral}, we followed the footsteps of \textbf{$\varphi$} and rushed into dimensional inflation. We saw how life establishes negative entropy enclaves and how civilizations cross technological singularities. That was a story about the \textbf{``Mind''}---the universe's will is wild, evolution is open, consciousness is the Red Queen's run. We were intoxicated by that dynamic vitality.

However, when these two pictures---the closed circle and the open spiral---are placed side by side, a profound tension arises.

Order and freedom, conservation and growth, cycle and ascension\ldots Are these two seemingly opposing concepts truly irreconcilable?

Is the underlying logic of the universe mechanical like a clock, or blindly growing like a vine?

To answer this question, we need a third key.

This key cannot only explain ``what the universe is'' (geometry), nor can it only explain ``where the universe goes'' (dynamics). It must explain \textbf{``why the universe exists.''}

This is the mission of the third book \textbf{The Natural Generator}.

In this book, we will excavate the third constant hidden at the deepest layer of mathematics---\textbf{$e$ (the natural constant)}.

It is the \textbf{``Spirit''} of the universe.

Why are physical laws always written in the form of exponential functions $e^{-iHt}$? This is by no means a mathematical coincidence.

$e$ possesses a unique divinity: \textbf{its derivative equals itself}. This means that in the logic of $e$, \textbf{existence (State)} is itself \textbf{propulsion (Trend)}. The universe needs no first mover, no external clockwork. The universe is a \textbf{self-referential, self-driven generative engine}.

In this book, we will dive into the deepest engine room of physics:

\begin{itemize}
\item Through \textbf{Euler's formula}, we will see how $\pi$ (circle) and $i$ (rotation) are unified by $e$.

\item Through \textbf{imaginary time} and \textbf{Wick rotation}, we will break the boundaries between quantum mechanics and thermodynamics, revealing the geometric truth that ``time is temperature.''

\item We will propose the \textbf{Modular Flow Hypothesis}, arguing that time is not external flow, but automatically ``secreted'' by the intrinsic entanglement structure of quantum states.
\end{itemize}

If the first two books describe the \textbf{``Form''} and \textbf{``Function''} of the universe, then this book will point directly to the \textbf{``Essence''} of the universe.

We will no longer be satisfied with observing that rotating vector; we will attempt to become the source of that rotation. We will see that all things are not created, but \textbf{Self-Generated}.

This is a pilgrimage to the ultimate ontology of physics. Let us abandon our attachment to ``beginning'' and ``ending,'' and enter that eternally generating exponential world.

\textbf{Haobo Ma}

December 2025, Singapore

