\section{Appendix A: Mathematical Derivation of Varying Constants}

In Chapter 2 of \textbf{Vector Cosmology IV}, we proposed a core viewpoint: the exponential growth of light speed $c$ will lead to the breaking of \textbf{Scale Invariance}, manifesting as tiny drift of the \textbf{Fine-Structure Constant ($\alpha$)}. The conclusions of that chapter were speculative; this appendix will provide a rigorous mathematical derivation framework.

We will show that in an expanding universe driven by $e$, the stability of fundamental constants is not absolute but depends on the \textbf{Geometric Shear} between growth rates of different physical sectors (electromagnetic, gravitational, quantum).

\subsection{A.1 Definition and Evolution of the Fine-Structure Constant}

The fine-structure constant $\alpha$ is the coupling constant of quantum electrodynamics (QED), determining the strength of electromagnetic interactions. Its standard definition is:

\[\alpha = \frac{e^2}{4\pi \epsilon_0 \hbar c}\]

In the standard model, all components ($e, \hbar, c, \epsilon_0$) are assumed to be absolute constants. But under the evolution equation of \textbf{FS Geometry}, we treat them as functions varying with intrinsic time $\tau$.

Taking the logarithm of both sides and differentiating:

\[\frac{\dot{\alpha}}{\alpha} = 2\frac{\dot{e}}{e} - \frac{\dot{\hbar}}{\hbar} - \frac{\dot{c}}{c}\]

(Assuming vacuum permittivity $\epsilon_0$ is a geometric normalization constant, not varying with time).

\subsection{A.2 Sector Growth Rates and Shear Factor}

According to our evolution equation, light speed $c(\tau)$ (representing the $v_{ext}$ sector or total budget) follows exponential growth:

\[\frac{\dot{c}}{c} = k_c = \frac{\ln \phi}{\pi}\]

However, charge $e$ and Planck constant $\hbar$ belong to the \textbf{$v_{int}$ (internal structure)} sector.

In an ideal ``synchronous expansion'' model, the internal sector should grow completely synchronously with the external sector, i.e., $k_e = k_\hbar = k_c / 2$ (based on dimensional analysis). At this point $\dot{\alpha} = 0$, scale invariance holds.

But the essence of \textbf{spiral geometry ($\phi$)} is non-resonant. This means there must be tiny \textbf{mismatches} between expansion rates of different dimensions.

We define the \textbf{Shear Factor ($\zeta$)} to quantify this mismatch:

\[2\frac{\dot{e}}{e} - \frac{\dot{\hbar}}{\hbar} = (1 - \zeta) \frac{\dot{c}}{c}\]

Substituting into $\alpha$'s evolution equation:

\[\frac{\dot{\alpha}}{\alpha} = (1 - \zeta) k_c - k_c = -\zeta k_c\]

\textbf{Conclusion:}

\begin{itemize}
\item If $\zeta = 0$ (perfect synchronization), $\alpha$ is constant.

\item If $\zeta \neq 0$ (shear exists), $\alpha$ will drift with time.
\end{itemize}

According to the \textbf{``Red Queen's Run''} theory, the expansion of external space ($v_{ext}$) is often slightly faster than the reorganization of internal structure ($v_{int}$). This means $\zeta$ is usually a tiny positive number ($\zeta \sim 10^{-5}$).

This causes $\alpha$ to \textbf{slowly decrease} with time.

\subsection{A.3 Physical Consequences: The Boundary of Atomic Stability}

What does the decrease of $\alpha$ mean for the material world?

Atomic binding energy (ionization energy) is roughly proportional to $\alpha^2 m c^2$.

If $\alpha$ decreases, the electromagnetic binding force between electrons and atomic nuclei will weaken.

When the cumulative drift of $\alpha$ reaches a certain critical threshold (e.g., $\Delta \alpha / \alpha \approx -4\%$), carbon nucleosynthesis (3$\alpha$ process) inside stars will be unable to resonate.

This not only means stars will extinguish but also means \textbf{the foundation of carbon-based life will be erased by physical laws}.

This provides a solid mathematical criterion for Chapter 5 ``The Carbon Sunset'': we must complete migration to light-based life before $\alpha$ drifts into the danger zone.

