\chapter{FS-Levinson Relation and Topological Counting}

In Chapter 7 ``The Holographic Pi Code'' of the main text, we defined the existence of matter as ``phase knotting'' in energy space. This profound physical picture is not fabricated out of thin air but is based on a geometric reconstruction of the famous \textbf{Levinson's Theorem} in classical scattering theory.

This appendix will briefly outline the mathematical proof of this relationship. We will show how to transform traditional scattering phase shift formulas into an inequality between \textbf{FS Geometric Length} and \textbf{Topological Winding Number} in projective Hilbert space. This provides solid mathematical support for the philosophical proposition that ``matter is counting.''

\section{Scattering Phase and Spectral Shift}

Consider a perturbed Hamiltonian $H = H_0 + V$, where $H_0$ is a free Hamiltonian with absolutely continuous spectrum, and $V$ is a potential term that decays sufficiently fast at infinity.

In scattering theory, the asymptotic behavior of the system is described by the scattering matrix $S(\omega)$. This is a unitary operator acting on the energy shell channel space. To extract topological information, we focus on its determinant:

\[\det S(\omega) = e^{i\phi(\omega)}\]

Here, $\phi(\omega)$ is defined as the \textbf{Total Scattering Phase}.

Mathematician M.G. Krein introduced the \textbf{Spectral Shift Function} $\xi(\omega)$ to describe the spectral difference between $H$ and $H_0$. Under appropriate conditions, the spectral shift and scattering phase have the following direct relationship:

\[\xi(\omega) = \frac{1}{2\pi}\phi(\omega) + n(\omega)\]

where $n(\omega)$ is an integer-valued function used to handle jumps when bound states cross energy thresholds.

\section{Topological Form of Levinson's Theorem}

Classical Levinson's theorem relates the phase shift at zero energy to the number of bound states. In the simplest case (no half-bound states, no threshold singularities), the theorem states:

\[\phi(0) - \phi(\infty) = N_b \pi\]

where:

\begin{itemize}
\item $\phi(0)$ is the phase in the zero-energy limit.

\item $\phi(\infty)$ is the phase in the high-energy limit (usually normalized to 0).

\item \textbf{$N_b$} is the total number of \textbf{Bound States} possessed by Hamiltonian $H$.

\item \textbf{$\pi$} is the circle constant.
\end{itemize}

This formula establishes the topological connection between ``discrete entities'' ($N_b$) and ``continuous variables'' ($\phi$) in physics.

\section{FS Geometric Length: From Topology to Geometry}

Now, we embed this relationship into \textbf{Fubini-Study Geometry}.

The mapping $\omega \mapsto \det S(\omega)$ defines a curve on the unit circle $U(1)$ in the complex plane.

The geometric properties of this curve can be described by the FS metric. On the $U(1)$ submanifold, the FS line element is proportional to the phase change rate $|\partial_\omega \phi(\omega)|$.

We define the \textbf{Total FS Length} of this curve from energy $\omega=0$ to $\omega=\infty$ as:

\[L_{FS} = \int_0^{\infty} \left| \frac{d\phi}{d\omega} \right| d\omega\]

In contrast, the \textbf{Topological Winding Number} (or total displacement) only cares about the difference between start and end points:

\[\Delta \Phi_{total} = |\phi(\infty) - \phi(0)| = N_b \pi\]

According to the integral inequality (the integral of absolute value is greater than or equal to the absolute value of the integral), or geometrically ``the straight line is the shortest path between two points'' (on a circle, it's the shortest arc length), we immediately obtain the \textbf{FS-Levinson Inequality}:

\[L_{FS} \ge N_b \pi\]

\section{Physical Meaning: The Cost of Knotting}

This inequality $L_{FS} \ge N_b \pi$ is the core formula in \textbf{Vector Cosmology} regarding the cost of matter's existence.

\begin{enumerate}
\item \textbf{Topological Lower Bound}:

    To ``create'' $N_b$ particles, the universe must complete at least $N_b$ full $\pi$ angle rotations in phase space. This is a hard index imposed by topology. Without sufficient phase winding, stable bound states cannot form.

\item \textbf{Geometric Efficiency}:

    \begin{itemize}
    \item If the scattering process is pure resonance (no background interference), the phase changes monotonically ($d\phi/d\omega$ does not change sign), then equality holds: $L_{FS} = N_b \pi$. This is the most efficient way to encode matter.

    \item If there is \textbf{negative time delay} or complex background scattering, the phase may locally backtrack. This leads to $L_{FS} > N_b \pi$. This means the universe pays extra geometric budget (takes a detour) to maintain the same particle number.
    \end{itemize}
\end{enumerate}

\section{Discreteness and Robustness}

Finally, under the discrete framework of QCA, this relationship still holds and is even more robust.

In lattice models, the energy spectrum is discrete, and the determinant $\det S(\omega_k)$ describes a polygonal path on the $U(1)$ circle.

We can calculate the \textbf{Discrete Winding Number} of this discrete path.

\begin{itemize}
\item This integer is not only well-defined but also insensitive to detailed perturbations of the lattice (topological protection).

\item It directly corresponds to the number of bound states within a finite volume.
\end{itemize}

Therefore, the FS-Levinson relation proves: \textbf{The ``granularity'' of matter (particle number) is essentially the ``number of loops'' of the holographic phase.} Whether in continuous field theory or microscopic QCA, the universe determines whether matter exists by ``counting loops.''

